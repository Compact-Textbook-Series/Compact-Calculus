\chapter*{Preface}
\addcontentsline{toc}{chapter}{Preface}

\section{Purpose of the textbook}

I probably made a lot of mathematicians angry by writing this book. But it misses their points. This textbook is designed for the practitioners of calculus: people that are interested in fields of sciences including biology, chemistry, and physics. Therefore, expect what you aren't expected, and don't expect what you traditionally expect.

The contents of this textbook are entirely self-contained and organized into four parts (volumes)\footnote{The exact number of parts may change as more material is added during the writing process.}: dynamical systems and ordinary differential equations, signal analysis, real and complex analysis, and vector calculus with partial differential equations. However, the topics deviates greatly from the traditional way calculus is taught. I took a great care of how the material is arranged in better needs practitioners, which means it may deviate from a standard calculus curriculum. To maintain focus within the core chapters, the auxilaries topics typically covered in traditional courses have been moved to the appendices.

Even though the book is very self-contained, it still doesn't serve as a standalone book. Sure, the story that I'm about to tell is quite fruitful, and you can enjoy it from the beginning to end. But, it lacks one thing: problem solving experiences. As of the revision today (\today), there are not that many exercises in the chapters. So, you might have to find some online. I've scattered plenty of them throughout the chapters.

\section{Layout of the book}

\subsection{Layout of Part I}

The first part of the book concerns differential equations: a powerful tool that can be used to describe how a system changes over time. The contents are laid out as follows:
\begin{itemize}
	\item \Cref{sec:derivatives,sec:integrals} develops the basic building blocks of calculus: derivatives and integrals, from the ground up
	\item \Cref{sec:basic-derivatives-and-integrals} takes the reader through derivatives and integrals of basic functions, e.g., exponentials, and logarithms.
	\item \Cref{sec:calculus-and-geometry} takes a detour from the main objective a bit, and looks at how calculus could be applied to solve various geometrical problems.
	\item \Cref{sec:calculus-and-trigonometry} studies the interaction between calculus and various trigonometric functions. It also applies calculus to describe some physical systems, specifically oscillatory ones. From this chapter, most of the foundations of calculus is already laid down, ready to be used.
	\item Finally, \cref{sec:calculus-and-physical-systems} uses all the knowledge developed in earlier chapters to explore many systems that exist in nature, including biological, chemical, and physical systems.
\end{itemize}
