\chapter*{Reading Guide}
\addcontentsline{toc}{chapter}{Reading Guide}

\textbf{Big disclaimer.} I am a physicist, and this is calculus from the physicist's point of view. So, there's going to be some contents from physics that pure mathematicians might not adore, I'm terribly sorry for that. 

The abstract contains the guide, the overview, and the mindset of the chapter. \emph{Reading the abstract is a necessity.}

The chapter contains the main idea of the topic. A chapter might have supplementary unnumbered chapters or sections which is optional.

Appendices either supplements or extend the chapter. Some even goes much beyond the chapter. They all vary. It's recommended to study them separately.

The interludes provide a \emph{historical background} for the next chapter. It's meant to connect one chapters with the other. It acts as a storyline bridge. If not taken, the next chapter might seem too terse. So, I recommend the reader to skim through the fruitfulness of historical development.

This book is separated into five parts
\begin{enumerate}[noitemsep, label = {\Roman*}]
    \item The fundamentals
    \item The applications
    \item The extensions
    \item The foundations, reimagined: real analysis
    \item Beyond imagination: complex analysis
\end{enumerate}

\emph{Part I (The fundamentals)} focuses on the basics of calculus: from derivatives to antiderivatives and some of its applications. I've swapped around the order of contents a lot as I see fit. I've also introduced some applications that's not found of normal pedagogy.

\emph{Part II (The applications)} focuses on further applications calculus to real world problems, mostly in physics.

\emph{Part III. (The extensions)} explores the realm of specialized calculus, most of the aren't even taught in universities: newer branches of calculus.

\emph{Part IV. (Real analysis)} and \emph{Part V. (Complex analysis)} as the name suggests, explores the full behaviors of real and complex functions. It reconsiders all the basics of calculus and dives deep into the backbone of all symbols that are abstracted away from the physical world.

At last, a fair warning; not all contents follow accurate historical order.