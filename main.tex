% !TEX program = xelatex
\documentclass[b5paper, 11pt, openleft]{memoir}

\usepackage{indentfirst}

%%% Font setup
\usepackage[no-math]{fontspec}
\setmainfont{ZillaSlab}[
    Path = fonts/,
    Extension = .ttf,
    UprightFont = *-Regular,
    ItalicFont = *-Italic,
    BoldFont = *-Bold,
    BoldItalicFont = *-BoldItalic,
    Ligatures = {TeX, Common}
]
\setmonofont[Scale = MatchLowercase]{Iosevka Custom}
\XeTeXlinebreaklocale="en_EN"
\XeTeXlinebreakskip=0pt plus 3pt
\emergencystretch=1em

\usepackage{mathtools}
\usepackage[math-style = ISO, mathrm = sym, warnings-off = {mathtools-colon}]{unicode-math}
\setmathfont[Scale = MatchLowercase]{Concrete-Math.otf}
\noDisplayskipStretch
\setoperatorfont\symscr
\everymath{\displaystyle}

%%% Stylings
% Page Layout and Margins
\setsecnumdepth{subsection}
\settocdepth{subsection}
\setlrmarginsandblock{4cm}{1.5cm}{*}
\setulmarginsandblock{3cm}{2.5cm}{*}
\setlength{\headheight}{30pt}
\setlength{\parindent}{1.5cm}
\setlength{\parskip}{0.3em}
\setlength{\beforechapskip}{20pt}
\renewcommand{\arraystretch}{1}
\allowdisplaybreaks
\setSpacing{1.5}
\checkandfixthelayout

% Footnotes
\usepackage{fancyhdr}
\pagestyle{fancy}
\fancyhead[LE]{\textbf{\thepage ~ $\big\vert$ \leftmark}}
\fancyhead[RO]{\textbf{\rightmark ~ $\big\vert$ \thepage}}
\fancyhead[RE, LO, C]{}

% Styling Figures
\usepackage{multicol, caption}
\setlength{\columnseprule}{1pt}
\def\columnseprulecolor{\color{lightgray}}
\DeclareCaptionLabelSeparator{pipe}{ $\vert$ }
\captionsetup{
    labelfont = {bf},
    font = {small, sc},
    width = 0.6\textwidth,
    labelsep = pipe,
    figurename = \textbf{Fig. }
}

% Styling Titles
\renewcommand{\partnamefont}{\LARGE\bfseries\scshape\centering}
\renewcommand{\partnumfont}{\LARGE\bfseries\scshape\centering\MakeUppercase}
\renewcommand{\midpartskip}{\par\rule{1in}{0.5pt}\vspace{1em}\par}
\renewcommand{\printparttitle}{\HUGE\bfseries\scshape\centering}
\renewcommand{\afterpartskip}{\relax}
\chapterstyle{veelo}
    \renewcommand*{\printchapternum}{%
    \makebox[0pt][l]{%
    \hspace{.8em}%
    \resizebox{!}{\beforechapskip}%
    {\chapnumfont \thechapter}%
    \hspace{.8em}%
    \rule{2\midchapskip}{\beforechapskip}%
    }%
}

% Packages
\usepackage[dvipsnames]{xcolor}
\usepackage{subcaption, graphicx, pdfpages, float, wrapfig}
\usepackage{csquotes}
\graphicspath{{diagrams/}}
\usepackage[inline]{enumitem}

\usepackage[colorlinks, allcolors = blue]{hyperref}
\usepackage{cleveref}

%%% Mathematical packages 
\usepackage[]{siunitx}
\usepackage{physics2}
\usepackage{derivative}
\usephysicsmodule{ab, ab.braket, ab.legacy, nabla.legacy, op.legacy}
\usepackage[makeroom]{cancel}
% Proofs
\usepackage{amsthm}
\usepackage{tcolorbox}
\tcbuselibrary{breakable, theorems, skins}
\newtcbtheorem[auto counter, crefname = {theorem}{theorems}, Crefname = {Theorem}{Theorems}]{theorem}{Theorem}{
    coltitle = black,
    sharp corners, frame hidden, enhanced, colback = lightgray!10, breakable,
    borderline west = {3pt}{-3pt}{lightgray},
    detach title = true,
    fonttitle = \bfseries, before upper = {\tcbtitle\quad}
}{theorem}
\newtcbtheorem[auto counter, crefname = {axiom}{axioms}, Crefname = {Axiom}{Axioms}]{axiom}{Axiom}{
    sharp corners, colback = lightgray!40, colframe = darkgray, breakable
}{axiom}
\newtcbtheorem[auto counter, crefname = {definition}{definition}, Crefname = {Definition}{Definition}]{df}{Definition}{
    sharp corners, colback = lightgray!40, colframe = darkgray, breakable
}{df}
\newtcbtheorem[auto counter, number within = section]{exmp}{Example}{
    colback = lightgray!40, colframe = darkgray, breakable
}{exmp}
\newtcbtheorem[auto counter, number within = chapter, crefname = {remarks of chapter }{remarks of chapter }, Crefname = {Remarks}{Remarks}]{remark}{Remarks on chapter }{
    colback = lightgray!10, colframe = black, breakable
}{remark}

%%% Mathematical commands
% Geometry
\let\line\overline
% Mathematical constants
\newcommand{\e}{\symrm{e}}
\newcommand{\im}{\symrm{i}}
\newcommand{\cpi}{\symrm{\pi}}
\DeclareMathOperator*{\ssum}{\symrm{\Sigma}}
\DeclareMathOperator*{\Proj}{\symrm{Proj}}
\DeclareMathOperator*{\sgn}{\symrm{sgn}}
% Vector notations
\newcommand{\vv}[1]{\pmb{\symrm{#1}}}
\newcommand{\vdot}{\pmb{\cdot}}
\newcommand{\conj}{^{\ast}}
\newcommand{\dagr}{^{\dag}}
\newcommand{\trnsp}{^{\intercal}}
\newcommand{\iden}{\symbb{I}}
\newcommand{\uv}[1]{\hat{\vv{e}}_{#1}}
\newcommand{\tensor}{\otimes}
\newcommand{\bmat}[1]{
	\begin{bmatrix}
		#1
	\end{bmatrix}
}
\newcommand{\ihat}{\hat{\i}}
\newcommand{\jhat}{\hat{\j}}
\newcommand{\khat}{\hat{k}}
% Differences
\DeclareMathOperator{\kdel}{\symrm{\delta}}
\DeclareMathOperator{\ddel}{\symrm{\delta}}
\newcommand{\Dd}{\symrm{\Delta}}
% Physics quantities symbols
\newcommand{\lagr}{\mathcal{L}}
\newcommand{\haml}{\mathcal{H}}
\newcommand{\hilb}{\mathcal{E}}
% Calculus notations
\newcommand{\appr}{\rightarrow}
\newcommand{\alc}[2][0.3]{&\parbox[c]{#1\textwidth}{#2}}
\newcommand{\pintm}[1]{\mathcal{D}[#1]}
% Mathematical conjunctions and expressions
\newcommand{\mathand}{\quad\textrm{and,}\quad}
\newcommand{\mathor}{\quad\textrm{or,}\quad}
\newcommand{\mathif}{\quad\textrm{if}\quad}
\newcommand{\mathiff}{\quad\textrm{\emph{iff}}\quad}
\newcommand{\maththerefore}{\therefore\emquad}
\newcommand{\ifft}{\emph{iff}}
% Notational commands
\newcommand{\flatfrac}[2]{#1\fracslash#2}
% Column types
\newcolumntype{C}{>{$}c<{$}}
\newcolumntype{L}{>{$}l<{$}}
\newcolumntype{R}{>{$}r<{$}}

%%% Type commands
\newcommand{\conclusion}{\section{Conclusion for Chapter \thechapter}}
\newcommand{\formula}{\section{Formula from Chapter \thechapter}}
\newcommand{\prelude}[1]{
    \chapter*{Prelude: #1}
    \addcontentsline{toc}{chapter}{Prelude: #1}
}
\newcommand{\prerequisites}[1]{\textbf{Prerequisites:}~\emph{#1}}

% Bibliographies
\usepackage[
    backend = biber,
    style = phys,
    sorting = anyvt
]{biblatex}
\addbibresource{bibliography.bib}

% Indices
\usepackage{imakeidx}
\makeindex

\begin{document}

\thispagestyle{empty}
\includepdf{coverpage.pdf}

\frontmatter

\prerequisites{set theory, algebra, geometry, basic trigonometry}

\tableofcontents

\chapter*{Preface}
\addcontentsline{toc}{chapter}{Preface}

\section{Purpose of the textbook}

I probably made a lot of mathematicians angry by writing this book. But it misses their points. This textbook is designed for the practitioners of calculus: people that are interested in fields of sciences including biology, chemistry, and physics. Therefore, expect what you aren't expected, and don't expect what you traditionally expect.

The contents of this textbook are entirely self-contained and organized into four parts (volumes)\footnote{The exact number of parts may change as more material is added during the writing process.}: dynamical systems and ordinary differential equations, signal analysis, real and complex analysis, and vector calculus with partial differential equations. However, the topics deviates greatly from the traditional way calculus is taught. I took a great care of how the material is arranged in better needs practitioners, which means it may deviate from a standard calculus curriculum. To maintain focus within the core chapters, the auxilaries topics typically covered in traditional courses have been moved to the appendices.

Even though the book is very self-contained, it still doesn't serve as a standalone book. Sure, the story that I'm about to tell is quite fruitful, and you can enjoy it from the beginning to end. But, it lacks one thing: problem solving experiences. As of the revision today (\today), there are not that many exercises in the chapters. So, you might have to find some online. I've scattered plenty of them throughout the chapters.

\section{Layout of the book}

\subsection{Layout of Part I}

The first part of the book concerns differential equations: a powerful tool that can be used to describe how a system changes over time. The contents are laid out as follows:
\begin{itemize}
	\item \Cref{sec:derivatives,sec:integrals} develops the basic building blocks of calculus: derivatives and integrals, from the ground up
	\item \Cref{sec:basic-derivatives-and-integrals} takes the reader through derivatives and integrals of basic functions, e.g., exponentials, and logarithms.
	\item \Cref{sec:calculus-and-geometry} takes a detour from the main objective a bit, and looks at how calculus could be applied to solve various geometrical problems.
	\item \Cref{sec:calculus-and-trigonometry} studies the interaction between calculus and various trigonometric functions. It also applies calculus to describe some physical systems, specifically oscillatory ones. From this chapter, most of the foundations of calculus is already laid down, ready to be used.
	\item Finally, \cref{sec:calculus-and-physical-systems} uses all the knowledge developed in earlier chapters to explore many systems that exist in nature, including biological, chemical, and physical systems.
\end{itemize}

\include{frontmatter/acknowledgement}
\chapter*{Reading Guide}
\addcontentsline{toc}{chapter}{Reading Guide}

\textbf{Big disclaimer.} I am a physicist, and this is calculus from the physicist's point of view. So, there's going to be some contents from physics that pure mathematicians might not adore, I'm terribly sorry for that. 

The abstract contains the guide, the overview, and the mindset of the chapter. \emph{Reading the abstract is a necessity.}

The chapter contains the main idea of the topic. A chapter might have supplementary unnumbered chapters or sections which is optional.

Appendices either supplements or extend the chapter. Some even goes much beyond the chapter. They all vary. It's recommended to study them separately.

The interludes provide a \emph{historical background} for the next chapter. It's meant to connect one chapters with the other. It acts as a storyline bridge. If not taken, the next chapter might seem too terse. So, I recommend the reader to skim through the fruitfulness of historical development.

This book is separated into five parts
\begin{enumerate}[noitemsep, label = {\Roman*}]
    \item The fundamentals
    \item The applications
    \item The extensions
    \item The foundations, reimagined: real analysis
    \item Beyond imagination: complex analysis
\end{enumerate}

\emph{Part I (The fundamentals)} focuses on the basics of calculus: from derivatives to antiderivatives and some of its applications. I've swapped around the order of contents a lot as I see fit. I've also introduced some applications that's not found of normal pedagogy.

\emph{Part II (The applications)} focuses on further applications calculus to real world problems, mostly in physics.

\emph{Part III. (The extensions)} explores the realm of specialized calculus, most of the aren't even taught in universities: newer branches of calculus.

\emph{Part IV. (Real analysis)} and \emph{Part V. (Complex analysis)} as the name suggests, explores the full behaviors of real and complex functions. It reconsiders all the basics of calculus and dives deep into the backbone of all symbols that are abstracted away from the physical world.

At last, a fair warning; not all contents follow accurate historical order.

\mainmatter

\part{The foundations}

\include{chapters/historyofcalculus}
\chapter{Writing down nature: derivatives}
\label{sec:derivatives}


\section{Invitation to calculus from a physicist}

\prerequisites{graphs, functions, kinematic variables}

\subsection{The mission of calculus}

Calculus is in two parts: differential and integral, which is linked by the fundamental theorem of calculus. Calculus \emph{links} physical systems to mathematical equations, so we can \emph{predict} the future of a system from the solution of those equations. But before we \emph{predict}, we must describe the system first: we must find a \emph{link}. And that's what we're going to do in this chapter.

The system that we'll use in this chapter is a mass that's traveling with some speed. We'll see how the attempt to describe this system naturally give birth to \emph{derivatives}. In the next chapter, we'll find the solution to the equation we've written. And at the end, we'll develop an intuition that
\begin{enumerate}[noitemsep]
    \item Derivatives measures the rate of change of a function w.r.t. a variable.
    \item Derivatives can be thought of the slope of a graph
    \item The universe is described in the language of differential equations
\end{enumerate}

\subsection{The notation of derivative calculus}
\label{sec:notation_of_calculus_derivative}

If $a$ is a variable, then $\dd{a}$ is a very small quantity of $a$ a.k.a. an \textbf{infinitesimal}. E.g., if $\vv{x}$ is displacement, then $\dd{\vv{x}}$ is a very small displacement. If $t$ is the time, then $\dd{t}$, a very short time.

We can also find ratio between infinitesimal. E.g., the ratio between some small distance and some short time, $\displaystyle\dv{\vv{x}}{t}$, we get the speed $\vv{v}$.

\subsection{Message from a physicist about the pedagogy}

Physics is about trying to understand how nature works. What we do in physics is that we first describe a system using mathematical equations, then use the solution of that equation to reveal the nature of that system. Here, I'd do the same.

If I were teaching a class, I'd tell you about car design. But, this is a textbook, and trying to deliver the same story here isn't that concise and cohesive. So, instead of following a story about cars, I'd like you to hop on with me and analyze the mechanics of a \textbf{speedometer}, which I reckon is the best introduction to calculus there is.

\section{Speed and instantaneous rate of change}
\label{sec:speedandinstantaneousrateofchange}

Let's analyze how a car \textbf{speedometer} work. If we're traveling at a speed $\qty{5}{\meter\per\second}$, when \emph{exactly} are we traveling at $\qty{5}{\meter\per\second}$? You could say $\vv{v} = \qty{5}{\meter}$ \emph{at the moment of measurement}. That's like saying, ``Oh, I can find the speed of the car by just taking a picture of it.'' But that's illegal! To calculate speed, we have to compare two points in space through time. Or, the rate of change of distance through time:
\begin{equation}
    \vv{v} = \frac{\vv{s}_2 - \vv{s}_1}{t_2 - t_1}. \label{eq:averagevelocity}
\end{equation}
While it may seem like cameras grab snapshots in an instant, they actually need time to take in light to construct an image, they need some \emph{exposure time} $\Dd{t}$.

To get a ``not blurred image'' of a moving object, we reduce the exposure time. If the exposure time is too long, the object will be smeared out. This effect is known as motion blur, which is normally undesired. But motion blur actually helps out a lot with measuring velocity. In which the velocity is just 
\begin{equation}
    \vv{v} = \frac{\textrm{Distance between the smear and the main object}}{\textrm{Exposure time}}.
\end{equation}
As illustrated in \cref{fig:car-motion-blur}, the motion blur clearly shows us the change in position of the car over the exposure time.
\begin{figure}[ht]
   \centering
   \includegraphics[width = \textwidth]{derivatives/motionblurredcar.pdf} 
   \caption{Calculating the velocity of a car from motion blur.}
   \label{fig:car-motion-blur}
\end{figure}

But what will happen with shorter exposure time? Does the motion blur disappear? No! The blur is still there, but it's just smaller. Typically, $\qty{12}{\ms}$ exposure time is short enough to create a ``focused image''. But it's just an illusion that came from the limitation of the screen's ability to reproduce such little blur. If our camera and screen is good enough, we can \emph{always} calculate the velocity of the car from the blur. The smaller the exposure time, the more detailed the image is, and the closer you'll get to the exact $\vv{v}$ at that moment in time. Finally, if we let the exposure time become infinitesimally short, we can say the $\vv{v}$ we got is \emph{the velocity at that exact point}. Or as we call it, the \textbf{instantaneous velocity}. By using the said calculus notation in \cref{sec:notation_of_calculus_derivative}, we can just write this as
\begin{equation}
    \vv{v} = \dv{\vv{x}}{t}, \label{eq:first_intro_to_derivative}
\end{equation}
and here it is ladies and gentlemen, the \textbf{derivative}: the measure of rate of change.

From here on out, I shall use \emph{derivative} and \emph{rate of change} interchangeably. So, every time you see \emph{derivative}, think \emph{rate of change}.

\section{An attempt to define derivatives}

Mathematically, a \textbf{derivative} is a measure of a function's rate of change with respect to a variable. In the previous example, $\vv{x}$ is a function that's dependent on time, and its derivative w.r.t. time, $\vv{v}$, is measuring the rate of change of function $\vv{x}(t)$ through time $\dd{t}$.

To find an explicit expression for derivative, let's say we have two points in spacetime $(\vv{x}_1, t_1)$ and $(\vv{x}_2, t_2)$. The change in $\vv{x}$ is $\vv{x}_2 - \vv{x}_1$, and the change in time is $t_2 - t_1$. The derivative of position w.r.t. time is then just
\begin{equation}
    \vv{v} = \dv{\vv{x}}{t} = \frac{\vv{x}_2 - \vv{x}_1}{t_2 - t_1}. \label{eq:derivative1}
\end{equation}
Let $t_2 = t + h$. Then, $\vv{x}_2 = \vv{x}(t + h)$ and $\vv{x}_1 = \vv{x}(t)$. The time difference used to calculate $\vv{v}$ must be miniscule: infinitesimally small. I have to introduce the notion of limits, which is just a fancier way of saying ``very close to, but not''
\begin{align}
    \vv{v} = \dv{\vv{x}}{t} &= \lim_{h \appr 0}\frac{\vv{x}(t + h) - \vv{x}(t)}{(t + h) - t} \\
    \vv{v} = \dv{\vv{x}}{t} &= \lim_{h \appr 0}\frac{\vv{x}(t + h) - \vv{x}(t)}{h}.
\end{align}
The equation above is what we generally refer to as the \emph{definition of derivative}. Then, we just extend this relation to any function $f(x)$, which requires just a substitution of variables. And then we get:
\begin{df}{Naive definition of derivative}{derivative_naive}
    \index{derivatives!naive definition}A derivative of a function $f(x)$ w.r.t. a variable $x$ is the rate of change of $f(x)$ w.r.t. $x$, and it is written as
    \begin{equation}
        \dv{f(x)}{x}~\textrm{or}~\dv{x}(f(x)),
    \end{equation}
    where
    \begin{equation}
        \dv{f(x)}{x} = \lim_{h \appr 0}\frac{f(x + h) - f(x)}{h}.\label{eq:naive_definition_of_derivative}
    \end{equation}
\end{df}
\Cref{eq:naive_definition_of_derivative} directly reads:
\begin{quote}
    The derivative of $f(x)$ with respect to $x$ is $\frac{f(x + h) - f(x)}{h}$ where $h$ is very close to $0$, but $h$ is strictly not $0$.
\end{quote}

\paragraph{On notation:} So far, we've been using the Leibniz's notation for derivatives, and it has the property that derivatives behave exactly like fractions, and you can cancel terms.
\begin{equation}
    \dv{a}{b}\times\dv{b}{c} = \dv{a}{c}.\label{eq:prechainrule}
\end{equation}
But, this isn't the only accepted notation. Multiple great mathematicians have come up with their own, and some are better than others in certain cases. I'd introduce other notations later on if necessary.

\section{The geometrical interpretation of the derivative}

\begin{wrapfigure}[15]{l}{0.4\textwidth}
    % \centering
    % \begin{subfigure}{0.45\textwidth}
        \centering
        \includegraphics{derivatives/positionversustime}
        \caption{Example of position vs. time graph where $\protect\vv{x}(t) = t^2$.}
        \label{fig:position_versus_time}
    % \end{subfigure}
    % \begin{subfigure}{0.4\textwidth}
    %     \centering
    %     \includegraphics{derivatives/positionversustime2}
    %     \caption{A ball traveling at constant velocity.}
    %     \label{fig:position_versus_time2}
    % \end{subfigure}
    % \caption{Various position versus time graph.}
\end{wrapfigure}

To interpret derivatives, notice that \Cref{eq:derivative1} looks a lot like the slope equation
\begin{equation}
    m = \frac{y_2 - y_1}{x_2 - x_1}. \label{eq:slope_equation}
\end{equation}
We've also seen that \cref{eq:derivative1} is analogous to the definition of derivative \cref{eq:naive_definition_of_derivative}. So is the derivative just the slope of a line? If it is, then what is the $y$-axis and the $x$-axis of a graph?

We can compare \cref{eq:slope_equation} to \cref{eq:derivative1}: the $y$-axis should be the position $\vv{x}$, and the $x$-axis, the time $t$. If we draw that out, we'll get the $x$-$t$ graph, which can encode the exact trajectory of an object. An example of which is shown in \cref{fig:position_versus_time}.

But what does this have to do with derivatives? \Cref{eq:slope_equation} only works for straight line! Well, here's the beauty of it. If you zoom into any points on a curve, eventually, it will look like a line. And thus, \emph{the derivative zooms into the curve at some point, and chooses two very close points on the curve and calculate its slope. In which, that slope represents the rate of change of the function at that point.}

\section{Evaluation of derivatives: method of increments}

Here, I'll give you an example of evaluating the derivative by using the definition, a.k.a. the \textbf{method of increments.}

In \cref{fig:position_versus_time}, the function $\vv{x}(t)$ is just $t^2$: the position at any time $t$ is $t^2$. I'd evaluate the velocity at $t = \qty{1}{\second}$. We start from \cref{eq:naive_definition_of_derivative}:
\begin{align*}
    \vv{v}(1) = \frac{\vv{x_2} - \vv{x_1}}{t_2 - t_1} &= \lim_{h \appr 0}\frac{\vv{x}(1 + h) - \vv{x}(1)}{(1 + h) - 1} \\
    &= \lim_{h \appr 0}\frac{(1 + h)^2 - 1}{h} \\
    &= \lim_{h \appr 0}\frac{2h + h^2}{h} \\
    &= \lim_{h \appr 0}2 + h.
\end{align*}

Since $h$ is very close to zero, we approximate $2 + h$ as $2$. Imagine comparing $2$ to $10^{-10}$. The $10^{-10}$ wouldn't make a noticeable difference, and we can ignore it. Therefore, $\vv{v}(t = \qty{1}{\second}) = \qty{2}{\meter\per\second}$.

Now, try evaluating $\vv{v}(t = \qty{3}{\second})$ for $\vv{x}(t) = t^3$. You should get $\qty{81}{\meter\per\second}$. As a hint, you can also ignore $h^2$ because if $h < 1$, then $h^2 < h$.

\section{Higher order derivatives}

\begin{wraptable}[13]{l}{0.35\textwidth}
    \begin{tabular}{c | c}
        Order & Name \\
        \hline
        1 & Velocity/Speed \\
        2 & Acceleration \\
        3 & Jerk \\
        4 & Snap/Jounce \\
        5 & Crackle \\
        6 & Pop
    \end{tabular}
    \caption{Higher order derivatives of position w.r.t. time}
    \label{tab:jerksnapcracklepop}
\end{wraptable}

\index{derivatives!higher order}
In kinematics, there are a whole set of quantities that can describe an object's trajectory, e.g., the acceleration, which is defined to be the rate of change of velocity w.r.t. time:
\begin{equation*}
    \vv{a}(t) = \dv{\vv{v}(t)}{t}.
\end{equation*}
But the $\vv{v}$ is also the rate of change of position w.r.t. time. Thus,
\begin{equation*}
    \vv{a}(t) = \dv{\vv{v}(t)}{t} = \dv{t}(\dv{\vv{r}(t)}{t}) = \dv[2]{\vv{r}(t)}{t}, 
\end{equation*}
where $\vv{r}$ is any position vector.

The $\dd[2]{\vv{r}}$ and $\dd{t}^2$ is just a matter of symbolic manipulation and should only be interpreted as just a shorthand. $\vv{a}$ is called the \textbf{second order derivative} of $\vv{r}$ because you've differentiated $\vv{r}$ twice. \textbf{Higher order derivatives} of position w.r.t. time is listed in \cref{tab:jerksnapcracklepop}.

\section{Expressing nature: basic differential equations}
\label{sec:basicdifferentialequations}

\begin{wrapfigure}[15]{r}{0.4\textwidth}
    \centering
    \includegraphics{derivatives/simplemotion}
    \caption{A ball dropped from height $h$}
    \label{fig:ball_dropped_from_a_height}
\end{wrapfigure}
The dynamics of a physical system is universally described by the famous Newton's second law $\vv{F} = m\vv{a}(t)$, which in derivatives form becomes:
\begin{equation}
    \vv{F} = m\dv[2]{\vv{r}(t)}{t}. \label{eq:newton_second_law}
\end{equation}
Let's try to describe a simple system with this. In \cref{fig:ball_dropped_from_a_height}, a ball is dropped from height $h$. The Earth's gravity pulls the ball with force $mg$ where $m$ is the mass of the ball, and $\vv{g}$, the acceleration from Earth's gravity. Newton's second law tells us that
\begin{align}
    \vv{F} &= m\dv[2]{\vv{r}(t)}{t} \nonumber\\
    m\vv{g} = m\dv[2]{\vv{r}(t)}{t} &\quad \textrm{or,} \quad \vv{g} = \dv[2]{\vv{r}(t)}{t},
\end{align}
which is sometimes called the \textbf{equation of motion}, which packs every information about this system you'd ever want. It directly reads as:
\begin{quotation}
    The acceleration of the ball is equals to $g$.
\end{quotation}
If you've stayed for this long, congratulations! You now have the power to describe every physical systems with mathematical equations using derivatives: the measure of the rate of change. But it might not be as useful as yet, just as a hammer may seem useless if used to paint, derivatives falls apart when you ask about the future of the system. E.g., how long the ball takes to reach the ground? Or, what's the position of the ball at a certain time? That's the job of the integral to solve, and we'll do so in the next chapter.

\conclusion

\begin{enumerate}[noitemsep]
    \item The concept of approaching can be used to bypass dividing by zero.
    \item Derivatives are rate of change of a function w.r.t. a variable which can be evaluated by the method of increments.
    \item Derivatives can be thought as the slope of a graph, or the tangent to a curve.
    \item Physical systems can be described by differential equations of different forms. One of them is the Newton's formulation stated in \cref{eq:newton_second_law}
\end{enumerate}

\begin{remark}{}{derivatives}
    \begin{enumerate}
        \item In \cref{sec:speedandinstantaneousrateofchange}, we zoomed in on the graph to approximate the function as a line. Actually, this is quite literally the whole idea of derivatives. If we dig in further in calculus, sometimes the rate of change analogy doesn't even make sense. However, saying that the derivative tries to approximate every function as a line works in all scenario. Though, it's quite abstracted away from the world.
    \end{enumerate}
\end{remark}
\chapter{Integrals and antiderivative}
\label{sec:integrals}

\prerequisites{\cref{sec:derivatives}, sigma summation notation}

Before we dive in, I shall clarify that every line, including straight, is a mathematical \index{curve}\textbf{curve}. You might see other textbooks use the term ``\emph{area under the graph}'' to refer to integrals. But, a \index{graph}\textbf{graph} is \textit{a diagram consisting of a line or lines, showing how two or more sets of numbers are related to each other} \cite{oxforddict}, not the curve itself. Therefore, I'll refer to a curve as any line that connects two points, whether straight or not. Now let's start.

\section{Invitation: mission impossible}

\subsection{The mindset of integral calculus}

In the previous chapter, we've learned how to describe the universe using derivatives. But derivatives falls short when we want to predict the future of a system. But, what do we mean when we say ``predict the system?''

In classical physics, a \textbf{state} represents the configuration that the system \emph{at one point in time}. To predict the system, we need to know the initial state of a system. Classical physics says that if you know the rules that the system plays by (In this case, $\vv{F} = m\odv[ord = 2]{\vv{r}}{t}$), and the initial condition, we can always determine the state of the system at any point in time. That is, classical physics guarantees that there is always a function, which takes in time as input, and outputs the state of a system. And in order to predict the system, we must know that function. But what may that function be?

Consider the example from \cref{sec:derivatives}. The equation of motion of the ball dropped from a height $h$ is a second-order differential equation
\begin{equation}
    g = \odv[ord = 2]{\vv{r}(t)}{t}, \label{eq:acceleration_from_gravity}
\end{equation}
with initial condition being $\vv{r} = h$. The function that describes the state is $\vv{r}(t)$, which outputs the position of the ball at a certain time $t$. You can see that this function is the derivative sign. To solve the differential equation for this function, we have to isolate it out, and undo the derivative sign. But how?

It seems impossible at first. We only know that $\vv{r}(t)$ must satisfy \cref{eq:acceleration_from_gravity}, i.e., the second derivative of $\vv{r}(t)$ is $g$. It's like we have to search through a gigantic pool of functions that mathematics have to offer to find a single function $\vv{r}(t)$ that satisfies \cref{eq:acceleration_from_gravity}. It'd be like finding a needle in the haystack!

Of course, this is a textbook, there must be a solution. If there is a will, there is a way. You might have to reverse engineer derivatives, which might look tedious at first. But well, you might find something interesting along the way.

\subsection{Brief notation of integral calculus}
\label{sec:brief_notation_of_calculus_integral}

The $\int$, a.k.a. the \textbf{integral}\footnote{Which also looks like a beansprout}, means to sum. This integral symbol is basically sigma summation symbol, but for infinitesimals. Therefore, we have bounds called \textbf{integral bounds}. E.g., if we sum a lot of little time step $\odif{t}$ together from $t_A$ to $t_B$ we get $t_B - t_A$, the total time step. Thus,
\begin{equation}
    \int_{t_A}^{t_B}\odif{t} = t_B - t_A.
\end{equation}

\section{Finding a function in the haystack}
\label{sec:function_in_the_haystack}

\Cref{eq:acceleration_from_gravity} has a second order derivative, let's go slowly and undo one derivative at a time. We'll undo the derivative of the RHS to get the velocity first. Then, we'll undo the derivative again to get the position.

\subsection{Step one: the velocity function from acceleration}

In \cref{eq:acceleration_from_gravity}, both $\vv{r}$ and $t$ is mixed up on the same side of the equation. That's not good for solving equations. So let's separate them. First, write $\odv[ord = 2]{\vv{r}(t)}{t}$ as $\odv{\vv{v}(t)}{t}$. Then, isolate $\vv{v}$ on one side and move $\vv{t}$ to the other.
\begin{align}
    \vv{g} &= \odv{\vv{v}(t)}{t} \\
    \odif{\vv{v}}(t) &= g\odif{t}.
\end{align}
This equation reads
\begin{quotation}
    \emph{A small change in velocity $\odif{\vv{v}}$ is product of $\vv{g}$ and a small time interval $\odif{t}$.}
\end{quotation}

\begin{multicols}{2}
To find the total change in velocity, we sum up a lot of small changes in velocity $\odif{\vv{v}}(t)$, which is equal to $\vv{g}\odif{t}$. Because $\odif{v}$ is directly proportional to $\odif{t}$, the total change in $\vv{v}$ is simply $\vv{g}t$. However, \emph{changes} doesn't say anything about the initial condition, so we add a term $C$ to compensate. Therefore,
\begin{equation}
    \vv{v}(t) = \vv{g}t + C.
\end{equation}
To find what $C$ is, just plug in the initial condition. If $\vv{v}(t = 0) = \vv{v}_0$, then
\begin{align}
    \vv{v}(t = 0) = v_0 &= \vv{g}\times 0 + C \\
    v_0 &= C.
\end{align}
Thus,
\begin{equation}
    \vv{v}(t) = \vv{g}t + \vv{v}_0.
\end{equation}
We can also express these ideas symbolically using the integral symbol (\cref{sec:brief_notation_of_calculus_integral}) as
\begin{align}
    \vv{g} &= \odv{\vv{v}(t)}{t} \\
    \odif{\vv{v}} &= \vv{g}\odif{t} \\
    \int\odif{\vv{v}} &= \int \vv{g}\odif{t} \\
    \vv{v} &= \vv{g}t + \vv{v}_0. \label{eq:simplemotionr}
\end{align}
\end{multicols}

\subsection{Step two: the position function from velocity}
Rewrite $\vv{v}$ as $\odv{\vv{r}}{t}$, then do separation of variables.
\begin{align}
    \vv{v} = \odv{\vv{r}}{t} &= \vv{g}t + \vv{v}_0 \label{eq:simplemotionr2} \\
    \odif{\vv{r}} &= \odif{t}(\vv{g}t + \vv{v}_0),\label{eq:simplemotionr3}
\end{align}
which reads,
\begin{quotation}
    \emph{A small change in position $\odif{\vv{r}}$ is the product of $\vv{g}t + \vv{v}_0$ and a small time interval $\odif{t}$.}
    \label{quote}
\end{quotation}
If we want to find the total change in position $\vv{r}$, we just have to sum all $\odif{\vv{r}}$'s. This time, it's not as obvious, because there's $\vv{g}t + \vv{v}_0$, which is also changing with time. So at each point in time, the rate of change of position is different.

\begin{figure}[ht]
    \centering
    \begin{subfigure}[b]{0.45\textwidth}
        \centering
        \includegraphics{integrals/simplemotionsubdivide}
        \caption{A subdivided graph}
        \label{fig:simplemotionsubdivided}
    \end{subfigure}
    \begin{subfigure}[b]{0.45\textwidth}
        \centering
        \includegraphics{integrals/simplemotionr}
        \caption{A graph showing the area of the gray trapezoid broken into two parts}
        \label{fig:simplemotionr}
    \end{subfigure}
    \caption{A $v$-$t$ graph of a ball dropped from a building}
\end{figure}

\begin{multicols}{2}
A good strategy in math if you don't know what to do is to just graph the function. The graph of \cref{eq:simplemotionr2} is shown in \cref{fig:simplemotionsubdivided}. A small time interval represents a little step in the $t$ axis. The curve shown in the graph represents $\vv{g}t + \vv{v}_0$. Therefore, $\odif{t}(\vv{g}t + \vv{v}_0)$ would just represent an area of a little rectangle $\odif{A}$ as shown in the figure.

The total change in position is the sum all those rectangles. When $\odif{t} \appr 0$, the sum of all $\odif{t}(\vv{g}t + \vv{v}_0)$ approaches the area under the graph, which can be calculated geometrically as shown in \cref{fig:simplemotionr}. Thus,
\begin{align}
    \vv{r} &= \frac{1}{2}(\vv{v}(t) - \vv{t}_0)t + \vv{v}_0t + C \\
    &= \frac{1}{2}(\vv{g}t - \vv{t}_0)t + \vv{v}_0t + C \\
    &= \frac{1}{2}\vv{g}\times t^2 + \vv{v}_0t + C
\end{align}
When $t = 0$, $\vv{r} = \vv{r}_0$. Therefore,
\begin{align*}
    \vv{r}_0 &= \frac{1}{2}\vv{g}\times 0^2 + \vv{v}_0\times 0 + C \\
    \vv{r}_0 &= C.
\end{align*}
Therefore,
\begin{equation}
    \vv{r}(t) = \frac{1}{2}\vv{g}t^2 + \vv{v}_0t + \vv{r}_0.
\end{equation}

To model the trajectory of the ball, we set
\begin{enumerate}[noitemsep]
    \item $\vv{v}_0 = 0$ (object is dropped and starts at zero speed)
    \item $\vv{r}_0 = 0$ (convenient initial condition placement)
\end{enumerate}
thus we get,
\begin{gather}
    \vv{r} = \frac{1}{2}\vv{g}t^2 \implies t = \sqrt{\frac{2\vv{r}}{\vv{g}}}
\end{gather}
\end{multicols}
Since the ground is at $\vv{r} = h$, the time that the ball hits the ground is then $\sqrt{2h/\vv{g}}$.

\subsection{Conclusion: area under the curve and antiderivative}

The form of the differential equation that we've solved is in the form $\odv{x}{t} = f(x)$. We have to undo the derivative, and the simplest way is to do separation of variables and turn the equation into
\begin{equation}
    \odif{x} = f(x)\odif{t},
\end{equation}
which reads,
\begin{quotation}
    A small change in $x$, i.e., $\odif{x}$ is represented by the area of a rectangle width $\odif{t}$ and height $f(x)$.
\end{quotation}
And, the total change $x$ is represented by the sum all those little rectangles, which is the area under the curve $f(x)$. Then, we add a constant $C$ to compensate for the initial condition. In symbolic form introduced in \cref{sec:brief_notation_of_calculus_integral}, it's just
\begin{align}
    \int\odif{x} &= \int g(x)\odif{t} \\
    x &= \int g(x)\odif{t}.
\end{align}

For now, we could say that integration is the reverse of derivatives. But to clearly see how this is linked for every function, we must study the fundamental theorem of calculus, which is the bridge between integration and differentiation.

Before we go there, let me clarify some terminologies. An \textbf{integral} refers to the area under the curve evaluated between two points. We say that an integral must have an \textbf{integral bound}. If we want to find the area under a function $f(x)$ from $x = a$ to $x = b$, we write it as
\begin{equation}
    A = \int_a^b f(x)\odif{x}.
\end{equation}
This just reads
\begin{quotation}
    The area $A$ under the curve $f(x)$ from $x = a$ to $x = b$ is equal to the sum of the area of many thin stripes width $\odif{x}$ height $f(x)$ that lies between $x = a$ and $x = b$.
\end{quotation}
The \textbf{antiderivative} however, refers to the function which takes in a value $a$, and output the integral of $f(x)$, evaluated from $0$ to $a$. Therefore, if $A(a)$ is the antiderivative of $f(x)$, then
\begin{equation}
    A(a) = \int_0^a f(x)\odif{x}.
\end{equation}
It also implies that if the area between $a$ and $b$ can be evaluated by
\begin{equation}
    \int_a^b f(x)\odif{x} = A(b) - A(a).
\end{equation}

\section{The fundamental theorem of calculus}
\label{sec:fundamentalthmofcalc}

\index{The fundamental theorem of calculus!first part}The intuition of fundamental theorem of calculus states that derivatives and integrals are essentially inverse of each other. In this section, I'll clarify this fact and make it more rigorous.
\begin{theorem}{The (first) fundamental theorem of calculus}{fundamentalthmofcalc1}
    If a function $f(x)$ has an antiderivative $A(x)$, then
    \begin{equation}
        \odv{A(x)}{x} = f(x).
    \end{equation}
\end{theorem}

\begin{figure}[b]
    \centering
    \includegraphics{integrals/fundamentalthmofcalc}
    \caption{The geometrical interpretation of part one of the fundamental theorem of calculus (\cref{theorem:fundamentalthmofcalc1}).}
    \label{fig:fundamentalthmofcalc}
\end{figure}
If $A(x)$ is the antiderivative of $f(x)$, then
\begin{equation}
    \int_{0}^{x} f(x)\odif{x} = A(x).
\end{equation}
The \emph{actual} area of one of the stripes (not rectangles) width $\odif{x}$ shown in \cref{fig:fundamentalthmofcalc}, it's obviously $A(x + \odif{x}) - A(x)$. The riemann sum approximation approximates the area by a small rectangle area $f(x)\odif{x}$. We can write the relation between the actual and the approximated area as
\begin{equation}
    A(x + \odif{x}) - A(x) \approx f(x)\odif{x}.
\end{equation}
To turn this into an equality, we add a correction term $\epsilon$
\begin{equation}
    A(x + \odif{x}) - A(x) = f(x)\odif{x} + \epsilon.
\end{equation}
If we let $\odif{x} \appr 0$, $\epsilon$ is negligible compared to $f(x)\odif{x}$; therefore,
\begin{equation}
    \lim_{\odif{x} \appr 0}A(x + \odif{x}) - A(x) = \lim_{\odif{x} \appr 0}f(x)\odif{x}.
\end{equation}
Since $f(x)$ is not a variable that's controlled by the limit sign,
\begin{align*}
    \lim_{\odif{x} \appr 0}A(x + \odif{x}) - A(x) &= f(x)\lim_{\odif{x} \appr 0}\odif{x} \\
    \lim_{\odif{x} \appr 0}\frac{A(x + \odif{x}) - A(x)}{\odif{x}} &= f(x).
\end{align*}
And here, we see that the L.H.S. is just the derivative of $A(x)$ w.r.t. $x$, thus
\begin{equation}
    \odv{A(x)}{x} = f(x),
\end{equation}
or ``\textit{the rate of change in area is the function itself}''. But the area function is given by the integral. This means
\begin{equation}
    \odv{x}\int f(x)\odif{x} = f(x):
\end{equation}
integrals and derivatives are inverses of each other. If we rephrase \cref{theorem:fundamentalthmofcalc1}, we see that ``\textit{the integral is the cumulative effect of the function}.''

\index{The fundamental theorem of calculus!second part}The second fundamental theorem of calculus,
\begin{theorem}{The second fundamental theorem of calculus (Newton-Leibniz rule)}{fundamentalthmofcalc2}
    If a function $f(x)$ has an antiderivative $A(x)$, then its indefinite integral from $a$ to $b$ is
    \begin{equation}
        \int_{a}^{b}f(x)\odif{x} = \int_{0}^{b}f(x)\odif{x} - \int_{0}^{a}f(x)\odif{x} = A(b) - A(a).
    \end{equation}
\end{theorem}
follows as a direct consequence of the geometrical interpretation of integrals: area under the curve.

The fundamental theorem of calculus allows us to evaluate integrals using derivatives. For example, if the derivative of $x^2$ is $2x$, then we also know that the integral of $2x\odif{x}$ is just $x^2$, which we'll discuss how to do that in the next chapter.

\section{How to calculate an integral? Riemann sum}
\label{sec:riemann_sum}

\begin{figure}[b]
    \centering
    \includegraphics{integrals/integralex}
    \caption{Illustration of Riemann sum of a function $f(x)$ from $0$ to $a$ by setting $k = 10$}
    \label{fig:integralex1}
\end{figure}

In \cref{sec:function_in_the_haystack}, we used geometry to find the area under a curve. However, that is not always possible, e.g., try integrating \cref{fig:integralex1} geometrically\footnote{This is what you'd get if you solve the simple harmonic oscillator}. It'd be impossible. But we can still approximate its area by slicing the area under the curve into thin rectangular stripes, then summing them. The approximated area is called the \index{Riemann sum}\textbf{Riemann sum}. Due to its computational cost, you don't really want to use this method. However, to develop a good intuition at the integral, we should still know its symbolic form.

Let there be a function $A(s)$ that represents the actual area of a function $f(x)$ from $0$ to $s$. The approximated area is then
\begin{equation}
    \sum_{i = 0}^{k}\odif{A_i} = \sum_{i = 0}^{k}\mathrm{width} \times \mathrm{height}
\end{equation}
where $k$ is the amount of subdivisions. From \cref{fig:integralex1}, the width of each stripe is $\flatfrac{a}{k}$, and the height of the $i$'th stripe is $f(\flatfrac{ia}{k})$. Therefore,
\begin{equation*}
    \sum_{i = 0}^{k}\frac{a}{k}\times f\left(\frac{ia}{k}\right).
\end{equation*}

In \cref{fig:integralex1}, we  $k = 10$. The area of the second rectangle $\odif{A_2}$ is
\begin{equation*}
    \odif{A_2} = \frac{a}{k}f\left(\frac{2a}{k}\right).    
\end{equation*}

For an arbitrarily finite $k$, the Riemann sum is just an approximation. If you want to find the \textit{actual} area under the curve, let $k \appr \infty$. The limit as $k \appr \infty$ is what we actually call the \textbf{integral}. Thus, we say
\begin{df}{Naive definition of integrals}{integralsnaivedefinition}
    The \index{integrals!definite integrals}(definite) integral, or the area under the curve of $f(x) = y$ from $0$ to $a$, is defined as
    \begin{equation}
        \int_{0}^{a}\odif{A} = \int_{0}^{a}f(x)\odif{x} = \lim_{k \appr \infty}\sum_{i = 0}^{k}\frac{a}{k} \times f\left(\frac{ia}{k}\right) = A(a).
    \end{equation}
    where $\int$ is the integral sign\footnote{Famously known for looking like a beansprout}. Here, $0$ is the lower bound of integration, and $a$, the higher bound. The function $A(x)$ is called the \textbf{antiderivative}, or the \index{integrals!indefinite integrals}\textbf{indefinite integral} of $f(x)$.
\end{df}

I shall put these definitions into perspective in the next two examples. It might use a bit of series knowledge. If you don't know, you can simply search up the summation identities that I'll use in \emph{Wikipedia} \cite{wikipedia-summation}.

\begin{exmp}{Riemann sum and antiderivative of $x^2$}{}
    Let $f(x) = x^2$, and let $A(a)$ be the antiderivative of $f(x)$, i.e., $A(a)$ is the area under the curve of $f(x)$ from $0$ to $a$. Note that
    \begin{equation}
        \sum_{i = 0}^{k}i^2 = \frac{k(k + 1)(2k + 1)}{6},
    \end{equation}
    The Riemann sum is then
    \begin{align}
        \sum_{i = 0}^{k}\frac{a}{k}&\times f\left( i a \over k \right) = \sum_{i = 0}^k\frac{a}{k}\times \left(ia\over k\right)^2 \\
        &= \sum_{i = 0}^k\frac{a^3}{k^3}\times i^2 \\
        &= \frac{a^3}{k^3}\times \frac{k(k + 1)(2k + 1)}{6}.
    \end{align}
    To find the antiderivative, let $k \appr \infty$. Notice that the $+1$ in the parenthesis are negligible when $k \appr \infty$. Therefore, we can write the antiderivative as
    \begin{align}
        A(a) &= \int_0^a f(x)\odif{x} \\
        &= \lim_{k \appr \infty}\frac{a^3}{k^3}\times \frac{k(k + 1)(2k + 1)}{6} \\
        &= \lim_{k \appr \infty}\frac{a^3}{k^3}\times \frac{2k^3}{6} \\
        &= \lim_{k \appr \infty}\frac{x^3}{3} = \frac{x^3}{3}.
    \end{align}
\end{exmp}

\begin{exmp}{Riemann sum and antiderivative of $x^3$}{}
    Let $f(x) = 4x^3$, and let $A(a)$ be the antiderivative of $f(x)$, i.e., $A(a)$ is the area under the curve of $f(x)$ from $0$ to $a$. Note that
    \begin{equation}
        \sum_{i = 0}^{k}i^3 = \left(\frac{k(k + 1)}{2}\right)^2,
    \end{equation}
    The Riemann sum is then
    \begin{align*}
        \sum_{i = 0}^{k}\frac{a}{k}&\times f\left( i a \over k \right) = \sum_{i = 0}^k\frac{a}{k}\times 4\left(ia\over k\right)^3 \\
        &= 4\sum_{i = 0}^k \left(\frac{x}{k}\right)^4 i^3 \\
        &= 4\left(\frac{x}{k}\right)^4\left(k(k+1)\over 2\right)^2.
    \end{align*}
    To find the antiderivative, let $k \appr \infty$. The $+1$ in $(k + 1)$ can be ignored as $k \appr \infty$. Therefore,
    \begin{align*}
        A(a) &= \int_0^a f(x)\odif{x} \\
        &= \lim_{k\appr\infty}4\left(x\over k\right)^4\left(k^2\over 2\right)^2 \\
        &= \lim_{k\appr\infty}x^4 = x^4.
    \end{align*}
\end{exmp}

\section{Basic applications of the integrals}

\subsection{Five equations of linear motion}
\label{sec:fiveequationsoflinearmotion}

\index{equation of motion!uniformed acceleration}So far, we have developed an intuition for derivatives, integrals, and their relationship. Let's apply those concepts to derive the famous equations of linear motion:
\begin{gather}
    v(t) = v_0 + at, \label{eq:linearmotion1}\\
    s = \frac{1}{2}(v_0 + v(t))t, \label{eq:linearmotion2}\\
    s = v_0(t)t + \frac{1}{2}at^2, \label{eq:linearmotion3}\\
    s = v(t)t - \frac{1}{2}at^2,\label{eq:linearmotion4}\\
    v(t)^2 = v_0^2 + 2as. \label{eq:linearmotion5}
\end{gather}

\begin{figure}[b]
    \centering
    \includegraphics{integrals/constantacceleration}
    \caption{$v$-$t$ graph of an object under constant or uniformed acceleration}
    \label{fig:vtconstantacceleration}
\end{figure}
Here, $v_0$ represents the initial velocity, $v(t)$, the velocity at any time $t$, $a$, the acceleration, and $s$, the displacement. These equations are derived from the Newton's second law on \emph{constant/uniformed acceleration} motion in one dimension, which we can evaluate using the geometrical interpretation of integrals developed earlier.

Because we've constrained the object to accelerate at $a$, the force exerted must be $ma$. Newton's second law $F = m\odv[ord = 2]{x}{t}$ then simplifies to $a = \odv[ord = 2]{x}{t}$. The same idea applies, we rewrite the equation in terms of $v$.
\begin{equation}
    a = \odv{v}{t} \label{eq:linearmotionintermediate}
\end{equation}
Which reads, ``\emph{the rate of change of $v$ w.r.t. $t$ is $a$}'' or ``\emph{the slope of the $v$-$t$ curve is always equals to $a$}''. Because $a$ is constant, the $v$-$t$ curve must be a straight line with slope $a$, shown in \cref{fig:vtconstantacceleration}.

\begin{multicols}{2}
\begin{proof}[Derivation of \cref{eq:linearmotion1}]
    We take advantage of the linearness of the curve. Just pick two points on \cref{fig:vtconstantacceleration}, as already shown, then
    \begin{align*}
        m = a &= \frac{\Delta v}{\Delta t} = \frac{v(t) - v_0}{t - 0} \\
        a & = \frac{v(t) - v_0}{t} \\
        v(t) &= v_0 + at.\qedhere
    \end{align*}
\end{proof}

\begin{proof}[Derivation of \cref{eq:linearmotion2}] 
    Use the reverse of \cref{theorem:fundamentalthmofcalc1}. We know that\footnote{Here, $x$ is replaced with $s$ to represent displacement in one dimension.} $v = \odv*{s}{t}$; therefore,
    \begin{equation*}
        \int v\odif{t} = s,
    \end{equation*}
    which reads ``\emph{The displacement $s$ is the area under the curve of a $v$-$t$ graph}''. From \cref{fig:vtconstantacceleration}, the area under the curve is a trapezoid with side length $v_0$, $v(t)$, and width $t$. Thus,
    \begin{equation*}
        s = \frac{1}{2}(v_0 + v(t))t,
    \end{equation*}
    which is just \cref{eq:linearmotion2}: the area of a trapezoid.
\end{proof}

\begin{proof}[Derivation of \cref{eq:linearmotion3,eq:linearmotion4,eq:linearmotion5}]
    We can arrange \cref{eq:linearmotion1} into three different ways, then plug in \cref{eq:linearmotion2}.
    First, $v(t) = v_0 + at$
    \begin{align*}
        s &= \frac{1}{2}(v_0 + v_0 + at)t \\
        &= v_0t + \frac{1}{2}at^2. 
    \end{align*}
    Second, $v_0 = v(t) - at$
    \begin{align*}
        s &= \frac{1}{2}(v(t) - at + v(t))t \\
        &= v(t)t - \frac{1}{2}at^2. 
    \end{align*}
    Third, $\displaystyle{t = \frac{v(t) - v_0}{a}}$
    \begin{align*}
        s &= \frac{1}{2}(v(t) + v_0)\frac{(v(t) - v_0)}{a} \\
        2as &= (v(t) + v_0)(v(t) - v_0) \\
        2as &= v(t)^2 - v_0^2 \\
        v(t)^2 &= 2as + v_0^2. \qedhere
    \end{align*}
\end{proof}
\end{multicols}

\subsection{The area of a circle}
\begin{figure}[t]
    \centering
    \includegraphics{integrals/dissectingcircle}
    \caption{(Left) Dissecting a circle radius $R$ radially into rings, each ring $\odif{r}$ thin. (Right) Stretching a ring into a trapezoid (Not to scale).}
    \label{fig:dissectingcircle}
\end{figure}

The perimeter of the circle is $2\cpi r$. For now, we only know how to find areas of polygons, but we want to know the circle's area. So, is there any way to turn a circle into a polygon? From \cref{sec:riemann_sum}, that the riemann sum can approximate areas under the curve using rectangular stripes. It'd be great if this circle can be turned into multiple rectangular stripes on a graph right?

So, we dissect a circle radially into small rings $\odif{r}$ thin, as shown in \cref{fig:dissectingcircle}. Then, stretch all the rings into a very thin trapezoid-like shape. It might seem impossible at first, but considering that our ring is very thin, it's quite easy to stretch it without breaking. I encourage you to grab a piece of paper, cut a really thin ring and try it out. If you actually do it, it'll look a bit warped. However, the warpedness will go away the thinner you go.

Since we sliced our trapezoid from a circle, for a stripe positioned at $r$, the inner side will be $2\cpi r$ long and the outer side, $2\cpi (r + dr)$. The area of the little trapezoid $\odif{A}$ then becomes
\begin{align*}
    \odif{A} &= \frac{1}{2}\odif{r}(2\cpi r + 2\cpi (r + dr)) \\
    &= 2\cpi r\odif{r} + 2\cpi\odif{r}^2. 
\end{align*}
As $\odif{r} \appr 0$, $\odif{r}^2$ becomes negligible. Therefore,
\begin{equation}
    \odif{A} = 2\cpi r \odif{r}.
\end{equation}
\begin{wrapfigure}[17]{r}{0.4\textwidth}
    \centering
    \includegraphics{integrals/dissectingcircle2}
    \caption{Rearranging all the approximated rectangles onto a graph (Not to scale).}
    \label{fig:dissectingcircle2}
\end{wrapfigure}
This equation says that for $\odif{r} \appr 0$, the trapezoid becomes a rectangle side length $\odif{r}$ and $2\cpi r$. Now, we have to sum it together. If we can put all these rectangles onto a graph, we can easily use the Riemann's sum to evaluate it. A natural way to do this is to put all the rectangles that we got from stretching the rings of the circle onto a graph one by one. The result would look something like \cref{fig:dissectingcircle2}\footnote{Not to scale}.

For every stripe at $r$, its height is $2\pi r$. If you were to plot the height of all rectangles when $\odif{r} \appr 0$, it'll eventually look like a curve that's given by $f(r) = 2\pi r$. We're interested in the area of the circle from $0$ to $R$. Now, it's transformed into the area under the curve of $f(r) = 2\pi r$: a triangle with base $R$ and height $f(R) = 2\pi R$. Thus, the area of a circle becomes
\begin{equation*}
    A = \frac{1}{2}R(2\pi R) = \pi R^2.
\end{equation*}

And there you go, you've turned essentially turned circle into a triangle and evaluate its area from there. I'd like to end off this chapter by mentioning the spirit of mathematics. Sometimes, you can't solve the problem directly. Most of the time, you have to re-frame the problem into another more-solvable problem. Problems like these often have the most sublime connections to the foundations of mathematics. This is a common theme in most of mathematics, especailly calculus. So, be sure to keep this in mind while reading through.

\conclusion

\begin{enumerate}
    \item Riemann sum are used to approximate areas under the curve of a function by using little rectangles then summing it.
    \item Integrals or anti-derivatives are functions that output the area under the graph of other functions.
    \item The limit where the width of the rectangles in the Riemann sum approaches zero, the Riemann sum becomes an integral.
    \item Integrals are the cumulative effect of a function.
    \item Integrals and derivatives are inverses of each other, and they're related by the fundamental theorem of calculus
    \item Integrals can be used in various ways by reframing questions into another simpler question.
\end{enumerate}
\chapter{Basic derivatives and antiderivatives}
\label{sec:basic_derivatives_and_integrals}

This chapter covers the derivatives and integrals of common functions: polynomials, exponential, and logarithms; focusing on their geometrical interpretation.

\prerequisites{binomial theorem (\cref{appendix:binomialexpansion}), basic trigonometry, derivatives, and integrals}

\paragraph{Terminologies:} The integral refers to the area under the curve. The antiderivative refers to the function $A(x')$ that outputs the area under the curve from $0$ to $x'$.

\section{Basic rules}
\label{sec:trivial_rules}

\index{derivatives!chain rule}
\paragraph{The chain rule} The Leibniz' notation treats derivative as fractions; thus, we can cancel terms (\cref{eq:prechainrule}). We exploit this property to find derivatives of composite functions. Consider two functions $f(x)$ and $g(x)$, the derivative of $f(g(x))$ can be found by a simple substitution. First let $g(x) = u$.
\begin{equation*}
	\odv*{f(g(x))}{x} = \odv{f(u)}{x}.
\end{equation*}
Let $1 = \odv{u}{u}$, then
\begin{equation*}
	\odv*{f(u)}{x} = \odv*{f(u)}{x} \times 1 = \odv{f(u)}{x} \odv{u}{u}.
\end{equation*}
Perform a change of denominator, then substitute back $u = g(x)$:
\begin{equation*}
	\odv*{f(g(x))}{x} = \odv{f(u)}{u} \times \odv{g(x)}{x}.
\end{equation*}
This is what we call the chain rule, or more generally
\begin{equation}
    \odv{y}{x} = \odv{y}{u} \times \odv{u}{x}. \label{eq:chainrule}
\end{equation}
where $u$ is a function of $y$.

The chain rule holds the intuition of how rate of changes relate to each other. E.g., the cheetah's speed is $10$ times the bicycle's speed, which is $4$ times the walking speed. The ratio between the cheetah's speed compared to walking speed would obviously be $10\cdot 4 = 40$:
\begin{equation}
    \odv{\textrm{Cheetah}}{\textrm{Walking}} = \odv{\textrm{Cheetah}}{\textrm{Bicycle}} \times \odv{\textrm{Bicycle}}{\textrm{Walking}}.
\end{equation}

\index{integral constant}
\paragraph{Integral constant} In \cref{sec:function_in_the_haystack}, each time we evaluate the antiderivative, we add an initial condition term, i.e., $v_0$ and $r_0$. So for any function $f(x)$,
\begin{equation}
    \int f(x)\odif{x} = A(x) + C
\end{equation}
where $C$ is any constant. However, \cref{theorem:fundamentalthmofcalc2} still holds for integrals with bounds.

\index{integrals!of an infinitesimal}
\paragraph{Integral of the infinitesimal} The antiderivative of the small rectangles $\odif{x}$ is just the total rectangle $x$ plus the integral constant $C$. 
\begin{equation}
    \int \odif{x} = x + C.
\end{equation}

\paragraph{Rules of equality} If two arguments are equal, their derivatives and antiderivatives w.r.t. the same variable must also be equal.
\begin{equation*}
    \textrm{If}~f = g,\textrm{ then }\odv{f}{x} = \odv{g}{x},\textrm{ and }\int f\odif{x} = \int g\odif{x} + C
\end{equation*}

\index{derivatives!of a constant}\paragraph{Derivative of a constant} A constant doesn't change; thus, the derivative of a constant is zero. 
\begin{equation}
    \odv*{(c)}{x} = 0.
\end{equation}

\section{Linearity of differentiation and integration}
\label{sec:diff_int_linearity}

\begin{figure}[ht]
    \centering
    \begin{subfigure}[t]{0.75\textwidth}
        \centering
        \includegraphics{basicderivativesandintegrals/derivativeconstantmultiple}
        \caption{for derivatives}
        \label{fig:derivativesconstantmultiple}
    \end{subfigure}
    \begin{subfigure}[b]{0.36\textwidth}
        \centering
        \includegraphics{basicderivativesandintegrals/integralconstantmultiple}
        \caption{for integrals}
        \label{fig:integralsconstantmultiple}
    \end{subfigure}
    \caption{Constant multiple rules}
\end{figure}

A \textbf{linear} mathematical entity are defined as anything that is compatible with addition and scaling. E.g., for a function $f(x)$ to be linear, it must obey
\begin{enumerate}[noitemsep]
    \item Additivity: $f(x + y) = f(x) + f(y)$
    \item Homogeneity of degree one: $f(ax) = af(x)$ for all constant $a$
\end{enumerate}
The simplest linear function there is, is a line that passes through the origin: $f(x) = ax$, in which
\begin{gather}
    f(x + y) = a(x + y) = ax + ay = f(x) + f(y), \\
    f(cx) = a(cx) = c(ax) = cf(x).
\end{gather}
Because this property, we associate a line with being linear. However, linearity doesn't have to always refer to lines.

Both derivatives and integrals are linear; for any function $f(x)$, $g(x)$, and constants $a$, $b$, the following rules follow.
\index{derivatives!constant multiple rule}\index{integrals!constant multiple rule}\paragraph{The constant multiple rule:}
\begin{align}
    \odv*{(af(x))}{x} &= a\odv{f(x)}{x}, &\alc[0.3]{Illustrated in \cref{fig:derivativesconstantmultiple}}\label{eq:derivativesconstantmultiple} \\
    \int af(x)\odif{x} &= a\int f(x)\odif{x}. &\alc[0.3]{Illustrated in \cref{fig:integralsconstantmultiple}}\label{eq:integralsconstantmultiple}
\end{align}
The geometrical interpretation of these two rules are very simple. When a function is multiplied by a constant $c$, its value is increased by a factor of $c$ everywhere. The slope must also be increased by $c$, and thus the function's area also.
\index{derivatives!sum rule}\index{integrals!sum rule}\paragraph{The sum rule:}
\begin{align}
    \odv*{(f(x) + g(x))}{x} &= \odv*{f(x)}{x} + \odv*{g(x)}{x}, \label{eq:derivativesumrule}\\
    \int f(x) + g(x)\odif{x} &= \int f(x)\odif{x} + \int g(x)\odif{x}. \label{eq:integralssumrule}
\end{align}
I.e., adding two functions increases its slope, thus also increasing its area under the graph.

\section{Derivatives and antiderivatives of polynomials}

Now that we've discussed the \enquote{trivial rules}, we're ready to tackle the easiest family of functions: the \textbf{polynomials}\index{polynomials}. They're in the form
\begin{equation}
    f(x) = a_0 + a_1x + a_2x^2 + a_3x^3 + \dots. \label{eq:polynomial_form}
\end{equation}
Let's focus on the antiderivative first; we'll see later how the antiderivative of polynomials can be found with just a simple substitution trick.

The form mentioned in \cref{eq:polynomial_form} is quite useless if we want to make progress. We can break it down by using the linearity of derivatives:
\begin{align}
    \odv{f(x)}{x} &= \odv*{(a_0 + a_1x^1 + a_2x^2 + a_3x^3 + \dots)}{x} \\
    &= a_1\odv{x^1}{x} + a_2\odv{x^2}{x} + a_3\odv{x^3}{x} + \dots.
\end{align}
Now we're left with derivatives of \emph{monomials} in the form of $x^n$. So let's do that instead.

\subsection{Derivatives of monomials: the power rule}
\label{sec:derivativespowerrule}

The method of increments allow us to quickly evaluate the derivative of $x^n$.
\begin{equation}
    \odv*{(x^n)}{x} = \lim_{h \appr 0}\frac{(x + h)^n - x^n}{h}.
\end{equation}
By using the binomial expansion (\cref{appendix:binomialexpansion}), 
\begin{align}
    &= \lim_{h \appr 0}\frac{\left(\sum_{k = 0}^n{n\choose k}x^{n - k}\cdot h^n\right) - x^n}{h} \\
    &= \lim_{h \appr 0}{{n\choose 0}x^nh^0 + {n\choose 1}x^{n - 1}h^1 + {n\choose 2}x^{n - 2}h^2 +\dots {n\choose n} + x^0h^n - x^n\over h} \\
    &= \lim_{h \appr 0}\frac{x^n + nx^{n - 1}h^1 + {n\choose 2}x^{n - 2}h^2 + \dots + h^n -x^n}{h} \\
    &= \lim_{h \appr 0}nx^{n - 1} + {n\choose 2}x^{n - 2}h + {n\choose 3}x^{n - 3}h^2 + h^{n - 1}
\end{align}
The terms with $h$ vanishes when $h \appr 0$; therefore,
\begin{equation}
	\odv*{(x^n)}{x} = nx^{n - 1}, \label{eq:power_rule}
\end{equation}
which is what we call the power rule. Because the binomial theorem work for all real numbers, this is true for any powers of $n$. But what about the geometrical interpretation given?

Let's now focus on the geometrical interpretation of the derivative of $x^2$: a function that represents the area of a square sidelength $x$. Its derivative then represents the ratio between the change in $x$, and the change of area when the sidelength is increased by $\odif{x}$.

\begin{figure}[b]
    \centering
    \includegraphics{basicderivativesandintegrals/xsquaredpowerrule}
	\caption{Interpretation of $\odv{x^2}{x}$ ($\odif{x}$ exaggerated)}
    \label{fig:xsquaredpowerrule}
\end{figure}
Illustrated in \cref{fig:xsquaredpowerrule},
\begin{equation}
    \odv*{(x^2)}{x} = \lim_{\odif{x} \appr 0} \frac{x\odif{x} + x\odif{x} + \odif{x}^2}{\odif{x}}. \label{eq:xsquaredpowerrule}
\end{equation}
When $\odif{x} \appr 0$, the $\odif{x^2}$ square would just become a single point. Compared to the big $x\odif{x}$ on the side, it's negligible; therefore, we can safely ignore it. The equation then becomes
\begin{align*}
    \odv*{(x^2)}{x} = \lim_{\odif{x} \appr 0}\frac{2x\odif{x}}{\odif{x}} = 2x,
\end{align*}
which is equivalent to the result from the power rule.

Deriving $\odv*{(x^3)}{x}$ geometrically shouldn't be too hard either. Take a cube sidelength $x$; increase its side length by $\odif{x}$, and compute the ratio between the change in volume and $\odif{x}$. The final answer should be $3x^2$. You could try with more higher order of $n$, but it'd be very hard to visualize.

Here I leave some exercises which shouldn't be too hard to do
\begin{enumerate}
    \item $\odv*{(x^2 - 2x + 16)}{x}$ \hfill $2x - 2$
    \item $\odv*{(x^3 + x^2 + x + 1)}{x}$ \hfill $3x^2 + 2x + 1$
    \item $\odv*{(3x^4 + 24x^3 - 2x^2 - 32x + 88)}{x}$ \hfill $12x^3 + 72x^2 - 4x - 32$
\end{enumerate}

\subsection{Antiderivatives of polynomials: the reversed power rule}

For the antiderivative of $x^n$, we use the fundamental theorem of calculus (\cref{theorem:fundamentalthmofcalc1}) on the power rule (\cref{eq:power_rule}),
\begin{equation*}
    x^n = \int nx^{n - 1} \odif{x}.
\end{equation*}
Substitute $n$ with $n + 1$
\begin{equation*}
    \int (n + 1)x^{n}\odif{x} = x^{n + 1},
\end{equation*}
and by the linearity of integrations (\cref{eq:integralsconstantmultiple}), we get the \index{integrals!reversed power rule}\textbf{reversed power rule}:
\begin{equation}
    \int x^{n}\odif{x} = \frac{x^{n + 1}}{n + 1}.
\end{equation}

Notice, this rule does not work for $n = 1$ because we can't divide by zero\dots \emph{or can we???} (Discussed further in \cref{sec:integralofthereciprocal})

\subsection{Extending the equations of linear motion}

Let's use the power rule to extend the equation of linear motions in \cref{sec:fiveequationsoflinearmotion} to include a constant jerk $j$, i.e.,
\begin{equation}
    \odv{a}{t} = j
\end{equation}
Then, we can move the $\odif{t}$ around and integrate both sides by using the reversed power rule.
\begin{align*}
    \int j\odif{t} &= \int\odif{a} \\
    j\int \odif{t} &= \int\odif{a} \alc[0.3]{Linearity of integrals} \\
    jt + a_0 &= a \alc[0.3]{Integral constant $a_0$}
\end{align*}
Because $a$ is the derivative of $v$, 
\begin{align*}
    \odv{v}{t} &= jt + a_0 \\
    \int \odif{v} &= \int jt + a_0 \odif{t} \\
    v &= j\int t\odif{t} + \int a_0\odif{t} \alc{Linearity of integrals} \\
    v &= \frac{1}{2}jt^2 + a_0t + v_0. \alc{Reversed power rule}
\end{align*}
We add $v_0$ as an integral constant in a similar manner. Since $v$ is the derivative of $r$,
\begin{align*}
    \odv{r}{t} &= \frac{1}{2}jt^2 + a_0t + v_0 \\
    \int \odif{r} &= \int\frac{1}{2}jt^2 + a_0t + v_0 \odif{t} \\
    r &= \frac{1}{2}j\int t^2\odif{t} + a_0\int t\odif{t} + v_0\int\odif{t} \\
    r &= \frac{1}{6}jt^3 + \frac{1}{2}a_0t^2 + v_0t + r_0.
\end{align*}
And there you have it! Further extensions can be made from here: the jerk could be time-dependent. But that's probably enough to illustrate my point. If you'd like to try, extend the equation of motion to have a constant snap: $\odv{j}{t} = s$. You should get
\begin{equation*}
    r = \frac{1}{24}st^4 + \frac{1}{6}jt^3 + \frac{1}{2}a_0t^2 + v_0t + r_0.
\end{equation*}

\section{Exponentials and growth}

\begin{wraptable}[15]{l}{0.445\textwidth}
    \centering
    \begin{tabular}{C | C | C}
        t & V(t) & V(t) - V(t - 1) \\
        \hline
        0 & 1 & \\
        1 & 2 & 2 - 1 = 1 \\
        2 & 4 & 4 - 2 = 2 \\
        3 & 8 & 8 - 4 = 4 \\
        4 & 16 & 16 - 8 = 8 \\
        5 & 32 & 32 - 16 = 16 \\
        6 & 64 & 64 - 32 = 32 \\
        7 & 128 & 128 - 64 = 64 \\
        8 & 256 & 256 - 128 = 128 \\
        9 & 512 & 512 - 256 = 256 \\
        10 & 1024 & 1024 - 512 = 12
    \end{tabular}
    \caption{Tables of $2^x$ plotted at interval $1$ from $0$ to $10$}
    \label{tab:exponentialbase2}
\end{wraptable}
The next function that we're going to discuss is exponentials: the mathematical representation of growth and decay. As an example, let's say there's a magical drop of water that doubles its volume $V$ every hour. I.e., for any time $t$,
\begin{equation}
    V(t + 1) = 2V(t). \label{eq:exponentialsrecurrencerelations}
\end{equation}
If the drop starts at one unit of volume, $V(0) = 1$; thus,
\begin{align*}
    V(1) &= 2V(0) = 2, \\
    V(2) &= 2V(1) = 2(2) = 2^2, \\
    V(3) &= 2V(2) = 2(2^2) = 2^3, \\
    V(4) &= 2V(3) = 2(2^3) = 2^4, \\
    &\vdots
\end{align*}
It's clear that the pattern is $V(t) = 2^t$: an exponential function. A natural question to ask is ``\emph{what is its rate of change?}''. Let's start by plotting the function over time (\cref{fig:exponentialgraph}). But, exponentials grow too quick to plot! By $V(7)$, we're already in the hundreds. So It'd probably be better to list the values of each point on a table.
\begin{figure}[t]
    \centering
    \includegraphics{basicderivativesandintegrals/exponentials}
    \caption{Exponential function $2^x$ plotted from $0$ to $8$}
    \label{fig:exponentialgraph}
\end{figure}
Notice that on the right most column of \cref{tab:exponentialbase2}, the difference between $V(t)$ and $V(t - 1)$ is exactly $V(t - 1)$: the function changes as much as its past-self. So does that mean that $\odv*{(2^x)}{x} = 2^x$?

Well sadly not, but close. See, \cref{tab:exponentialbase2} only shows a \emph{discrete} step. You can write it out as
\begin{equation}
    \frac{2^{x + 1} - 2^x}{1} = 2^x\left(\frac{2 - 1}{1}\right) = 2^x,
\end{equation}
that is why $V(t) - V(t - 1) = V(t - 1)$. But we still need the method of increments to calculate the derivative of $2^x$:
\begin{align*}
    \odv*{(2^x)}{x} &= \lim_{h \appr 0}\frac{2^{x + h} - 2^x}{\odif{x}} \\
    &= 2^x\lim_{h \appr 0}\left(\frac{2^h - 1}{h}\right).
\end{align*}
Now you could try plugging in a really small value of $h$, say $0.000001$. The term $\frac{2^h - 1}{h}$ will approach $0.69314\dots$. If you try other bases of exponents, say $3$, you might see a pattern emerging.
\begin{equation}
    \odv*{(3^x)}{x} = 3^x\lim_{h \appr 0}\frac{3^h - 1}{h}.
\end{equation}
\emph{The rate of change of an exponential function is always itself times a proportionality constant}. For $3^x$, it's about $1.09851\dots$. If we could find a number $n$ where $\frac{n^h - 1}{h} = 0$, we'd have a very pretty function which it is its own derivative. So let's find that!
 
\subsection{A function that is its own derivative}
\label{sec:function_equals_own_derivative}

Let's set a goal: find the function that is its own derivative. I shall introduce a substantial concept in calculus: the expansion of functions. Every function has a polynomial expansion\footnote{Although the convergence of the series derived is quite questionable; thankfully, the power series of $n^x$ converges everywhere.} called the \index{power series!introduction}\textbf{power series}. For every $f(x)$,
\begin{equation}
    f(x) = a_0 + a_1x^1 + a_2x^2 + a_3x^3 +\dots. \label{eq:naivetaylorseries}
\end{equation}
E.g., $\sin(x)$ can be written as
\begin{equation}
    \sin(x) = x - \frac{1}{3!}x^3 + \frac{1}{5!}x^5 - \frac{1}{7!}x^7 + \dots. \label{eq:taylorseriessine}
\end{equation}
we will derive this expression later in \cref{sec:sine_cosine_power_series}. For now, we can just use \cref{eq:naivetaylorseries} to find the expression for the function that is its own derivative.

We've seen that the exponential is a possible candidate for a function that is its own derivative. Now, assume that for some real number $n$,
\begin{equation}
    \odv*{(n^x)}{x} = n^x.
\end{equation}
Then, we use the polynomial expansion and the power rule,
\begin{align*}
    \odv*{(a_0 + a_1x^1 + a_2x^2 + a_3x^3 + \dots)}{x} &= a_1 + 2a_2x^1 + 3a_3x^2 + 4a_4x^3 + \dots \\
    &= n^x
\end{align*}
If the function is its own derivative, the polynomial expansion of the function and its derivative must be the same.
\begin{gather}
    n^x = a_0 + a_1x^1 + a_2x^2 + a_3x^3 + \dots, \label{eq:naivetaylorseries1}\\
    n^x = a_1 + 2a_2x^1 + 3a_3x^2 + 4a_4x^3 + \dots. \label{eq:naivetaylorseries2}
\end{gather}
Since both are polynomials, we can match the coefficient here:
\begin{equation}
    \begin{array}{c}
    a_0 = a_1 \\ a_1 = 2a_2 \\ a_2 = 3a_3 \\ a_3 = 4a_4 \\ \vdots
    \end{array}\label{eq:naivetaylorserieserelations}
\end{equation}
$a_0$ and $a_1$ is relatively easy to find. As we've seen, $n$ must be between $2$ and $3$. By the properties of exponentials, $x = 0 \implies n^x = 1$. We can then plug $x = 0$ and set $n^x = 1$ into \cref{eq:naivetaylorseries2}:
\begin{align*}
    1 &= a_0 + a_1(0)^1 + a_2(0)^2 + a_3(0)^3 + \dots \\
    a_0 &= 1.
\end{align*}
Since $a_0 = a_1$, $a_1$ must also be $1$. We can then go back to \cref{eq:naivetaylorserieserelations} and get
\begin{equation*}
    n^x = 1 + 1 + \frac{1}{2!}x + \frac{1}{3!}x^3 + \frac{1}{4!}x^4 + \dots
\end{equation*}
The pattern here is clearly $a_n = n!$. If we want to find $n$, we just let $x = 1$.
\begin{equation*}
    n = 1 + 1 + \frac{1}{2!} + \frac{1}{3!} + \frac{1}{4!} + \dots,
\end{equation*}
and there we have an expression for $n$ which is an irrational number. If you work this out, it's around $2.71828\dots$. Because $(2.71828\dots)^x$ is its own derivative, it's very useful in mathematics and appears everywhere, even at the seams of mathematics that doesn't even seems related to growths: the patterns of prime number, this constant $2.71828\dotso$ has a name and symbol: the Euler's number\footnote{Not to be confused with the ``Euler's constant'' which is another constant written $\gamma$, and is around $0.57721\dots$}, written as $\e$ where
\begin{equation}
    \e = \sum_{i = 0}^{\infty}\frac{1}{i!}
\end{equation}

\subsection{\protect Another interpretation of $\e$: infinite bank interests}

There are two types of bank interests: simple and compound. Simple interests is the thing that you don't really want: the interest is always the same and doesn't grow with your account. You can calculate it by using
\begin{equation}
    n(t) = n_0 + tr
\end{equation}
where $n(t)$ is the total money at time $t$, $n_0$ the initial money in your bank account, and $r$, the interest rate.

Compound interest in the other hand calculates your interest based on how much money you have at that moment:
\begin{equation}
    n(t + 1) = n(t)r + n(t).
\end{equation}
We can find the expression for $n(t)$ in a similar fashion to what we've done in \cref{eq:exponentialsrecurrencerelations}. You'll get
\begin{equation}
    n(t) = (1 + r)^tn_0 \label{eq:compoundinterestformula}
\end{equation}
which is an exponential function.

Let's say you deposit $100\$$ into a bank and the bank is offering you two options on \index{compound interest}\textbf{compound interests} rate. \begin{enumerate*}[label = \arabic*)]\item Take $100\%$ interest in $1$ year, \item Take $\flatfrac{100}{2}\%$ twice a year, or \item Take $\flatfrac{100}{356}\%$ daily.\end{enumerate*} If you take option one, you'd end up with $200\$$. Option two takes you to $225\$$, and option three takes you to around $271.447\dots\$$. You might see a theme here. If you get $\flatfrac{100}{n}\%$ interest, $n$ times a year, the result keeps getting higher. Is there an upper limit to this?

If we write it in terms of limits as $n \appr \infty$ and use \cref{eq:compoundinterestformula}, the compound interests formula,
\begin{equation*}
    x = \lim_{n \appr \infty}\left(1 + \frac{1}{n}\right)^nn_0.
\end{equation*}
Where $x$ is the total money after a year. We're interested in the upper limit, so we'll just let $n_0$ for now. The expression will become
\begin{equation}
    x = \lim_{n \appr \infty}\left(1 + \frac{1}{n}\right)^n \label{eq:infinitecompoundinterest}
\end{equation}
Technically, we could go in and substitute a very high $n$, such as $1000000$. But I believe you could already see that it would be a nightmare to calculate: exponentiation is not at all an easy task. However, notice that from option three earlier, the total money is $271.447\dots\$$ which is suspiciously similar to $\e$ at $2.71828\dots$. If \cref{eq:infinitecompoundinterest} equals \cref{eq:naivetaylorserieserelations}, we'd find the upper limit for this problem and solve the mystery.

We can use the binomial theorem on \cref{eq:infinitecompoundinterest} and get
\begin{align*}
    &\lim_{n \appr \infty}\sum_{k = 0}^{n}{n \choose k}1^{n - k}\frac{1}{n^k} \\
    &= \lim_{n \appr \infty}{n \choose 0}\frac{1}{n^0} + {n \choose 1}\frac{1}{n^1} + {n \choose 2}\frac{1}{n^2} + {n \choose 3}\frac{1}{n^3} + \dots.
\end{align*}
Then, we use the definition of $n$ choose $k$,
\begin{align*}
    &= 1 + \lim_{n \appr \infty}\frac{n!}{1!(n - 1)!}\frac{1}{n^1} + \frac{n!}{2!(n - 2)!}\frac{1}{n^2} + \frac{n!}{3!(n - 3)!}\frac{1}{n^3} + \dots.
\end{align*}
Now, we can cancel the $n!$ on the numerator to the denominator and isolate the factorials.
\begin{align*}
    &= 1 + \lim_{n \appr \infty}\frac{n(n - 1)!}{(n - 1)!}\frac{1}{1!n^1} + \frac{n(n - 1)(n - 2)!}{(n - 2)!}\frac{1}{2!n^2} + \dots \\
    &= 1 + \frac{1}{1!} + \lim_{n \appr \infty} \frac{n(n - 1)}{2!n^2} + \frac{n(n - 1)(n - 2)}{3!n^3} + \frac{n(n - 1)(n - 2)(n - 3)}{4!n^4} + \dots
\end{align*}
Notice, as $n \appr \infty$, the ratio between $n + R$ and $n$ where $R$ is any real numbers would be literally negligible. For every terms in our series, both the numerator and the denominator has the same polynomic degrees. Therefore, all the $n$'s in the series cancel out and we get
\begin{equation}
    x = 1 + \frac{1}{1!} + \frac{1}{2!} + \frac{1}{3!} + \dots
\end{equation}
which is literally \cref{eq:naivetaylorserieserelations}; thus,
\begin{equation}
	\e = \lim_{n \appr \infty}\ab(1 + \frac{1}{n})^n. \label{eq:e-def-infinite-interests}
\end{equation}

In conclusion, the upper limit that the bank can give you is $\e$, which should make sense geometrically. Because we're gradually turning a discrete interest into a continuous one, $\e$ should appear in the limit of the continuous bank interests.

The limit definition of $\e$ allows us to express $\e^x$ in terms of limits.
\begin{align}
	\e^x &= \ab(\lim_{n \appr \infty}\ab(1 + \frac{1}{n})^n) \\
		 &= \lim_{n \appr \infty}\ab(1 + \frac{1}{n})^{nx}
\end{align}
If $n = \flatfrac{u}{x}$, then
\begin{align}
	\e^x &= \lim_{\frac{u}{x} \appr \infty}\ab(1 + \frac{u}{h})^{h}
\end{align}
For $\frac{u}{x} \appr \infty$ to be true for all finite $x$, $u$ must approach $\infty$; therefore,
\begin{align}
	\e^x &= \lim_{u \appr \infty}\ab(1 + \frac{u}{h})^h
\end{align}

\subsection{The antiderivative of exponential functions}

It should be trivial that if $\e^x$ is the derivative of itself, so is its antiderivative
\begin{equation}
    \int \e^x \odif{x} = \e^x + C.
\end{equation}
For other bases, we could use the fundamental theorem of calculus (\cref{theorem:fundamentalthmofcalc1}) to find its antiderivative. I.e., if
\begin{equation*}
    \odv{(n^x)}{x} = n^x\lim_{h \appr 0}\frac{n^h - 1}{h},
\end{equation*}
then
\begin{align*}
    \odv*{(n^x)}{x} &= n^x\lim_{h \appr 0}\frac{n^h - 1}{h} \\
    n^x &= \lim_{h \appr 0}\frac{n^h - 1}{h}\int n^x\odif{x} \\
    \int n^x\odif{x} &= n^x\lim_{h \appr 0}\left(\frac{n^h - 1}{h}\right)^{-1}.
\end{align*}
The term in the limit sign still appears here. If we want to uncover how this term comes to be, we must discuss the logarithms.

\section{Logarithms}

Logarithms are inverses of exponential, like how roots are inverses of monomials. To illustrate what I mean,
\begin{equation*}
	\textrm{For monomials, } \sqrt[a]{x^a} = x, \textrm{ but with logarithms, } \log_{a}(a^x) = x.
\end{equation*}
They have the following properties:
\begin{gather}
    \log_a(x) + \log_a(y) = \log_a(xy), \\
    \log_a(x) - \log_a(y) = \log_a\left(\frac{x}{y}\right), \\
    \log_a(x) = \frac{\log_b(x)}{\log_b(a)}.
\end{gather}
Since $\e^x$ is shown to be a very important function in modelling continuous growth, and is its own derivative, we give its inverse function its own name: the \index{logarithms!natural logarithms}\textbf{natural logarithm}, written as $\ln(x)$.

What's the derivative of $\ln(x)$? By the method of increments,
\begin{align}
	\odv*{\ln(x)}{x} &= \lim_{h \appr 0}\frac{\ln(x + h) - \ln(x)}{h} \\
					&= \lim_{h \appr 0}\frac{\ln(\frac{x + h}{x})}{h} \\
					&= \lim_{h \appr 0}\frac{\ln(1 + \frac{h}{x})}{h} \\
					&= \lim_{h \appr 0}\ln\ab(\ab(1 + \frac{h}{x})^{\frac{1}{h}}).
\end{align}
This form is very similar to the limit definition of the exponential function $\e^x$, but with some terms swapped around. So let's try to fit this limit to the form of $\e^x$. I shall let $h = \frac{1}{n}$. If $h \appr 0$, then $n \appr \infty$. Therefore,
\begin{align}
	\odv*{\ln(x)}{x} &= \lim_{n \appr \infty}\ln\ab(\ab(1 + \frac{1}{nx})^{n}) \\
					 &= \lim_{n \appr \infty}\ln\ab(\ab(1 + \frac{\flatfrac{1}{x}}{n})^n). \label{eq:ln-diff-derivation}
\end{align}
Recall that
\begin{equation}
	\e^x = \lim_{n \appr \infty}\ab(1 + \frac{x}{n})^n.
\end{equation}
Thus,
\begin{equation}
	\lim_{n \appr \infty}\ab(1 + \frac{\flatfrac{1}{x}}{n})^n = \e^{\frac{1}{x}}.
\end{equation}
which fits the form of \cref{eq:ln-diff-derivation}. Therefore,
\begin{align}
	\odv*{\ln(x)}{x} &= \lim_{n \appr \infty}\ln(\e^{\frac{1}{x}}) \\
					 &= \lim_{n \appr \infty}\frac{1}{x} = \frac{1}{x}:
\end{align}
the derivative of the natural logarithm of $x$ is the reciprocal of $x$.

This result can also be interpretted geometrically. If the exponential grows very quickly when $x$ increases, its inverse must grow slow. The slow growing rate is directly captured by the function $\frac{1}{x}$.

Here is the perfect time that I'll give you a glimpse into calculus with many variables: multivariable calculus. I shall present another way to derive the derivative of $\ln(x)$: \textbf{implicit differentiation}.\index{implicit differentiation}

Functions in the form $f(x) = y$ are called \textbf{explicit functions}: the relations between the two variables are written explicitly. Functions that can't be written as $f(x) = y$ are called \textbf{implicit functions}, e.g., $x^2 + y^2 = r^2$. \textbf{Implicit differentiation} concerns the differentiation of implicit functions. In this case, it actually makes our life easier if we turn $\ln(x)$ into an implicit function.

If I let $\ln(x) = y$, I can raise both sides to the power of $\e$ and get
\begin{equation}
    \e^{\ln(x)} = \e^y.
\end{equation}
Because logarithms are inverses of exponential,
\begin{equation}
    x = \e^y.
\end{equation}
By the rule of equality, (\cref{sec:trivial_rules}), we can take the derivative of both sides w.r.t. $y$ instead of $x$:
\begin{align*}
    \odv{x}{y} &= \odv{\e^y}{y} \\
    \odv{x}{y} &= \e^y.
\end{align*}
But we're looking for the derivative of $y$ (which is just $\ln(x)$) w.r.t. $x$, not the derivative of $x$ w.r.t. $y$. Here's where Leibniz's notation comes into clutch: we can swap the numerator with the denominator for both sides then, substitute back $y = \ln(x)$:
\begin{align*}
    \odv{\ln(x)}{x} &= \frac{1}{\e^{\ln(x)}} \\
    \therefore~\odv{\ln(x)}{x} &= \frac{1}{x},
\end{align*}
and there: the derivative of the natural logarithm is the reciprocal.

With the power of natural logarithms, we can actually go back at the derivative of $n^x$ and finally uncover the mystery behind the proportionality term that's lingering around. Start with the manipulation of $n^x$.
\begin{equation}
    n^x = \left(\e^{\ln(n)}\right)^x = \e^{x\ln(x)}.
\end{equation}
With the chain rule (\cref{sec:trivial_rules}), let $u = x\ln(n)$:
\begin{align*}
    \odv*{(n^x)}{x} &= \odv{\e^{x\ln(n)}}{x} \\
    &= \odv{\e^u}{x} \times \odv{u}{u} \\
    &= \odv{\e^u}{u} \times \odv{u}{x} \\
    &= \e^{x\ln(n)} \times \odv{x\ln(n)}{x} \\
    &= n^x\ln(n).
\end{align*}
And here it is: the mystery proportionality constant. It's just a consequence of the natural logarithm. Thus, one way to define the natural log would be
\begin{equation}
    \ln(n) = \lim_{h \appr 0}\frac{n^h - 1}{h}.
\end{equation}
The antiderivative of other bases exponents are then given by
\begin{equation}
    \int n^x = \frac{1}{\ln(n)}n^x + C.
\end{equation}

\subsection{The product rule and the quotient rule}
\index{derivatives!product rule}

Sometimes, we have to multiply the two functions together before taking the derivatives. There are two ways to do this. To keep the spirit of visualization, I shall first introduce the geometrical way, then the analytical way.

\begin{figure}[h]
    \centering
    \includegraphics{basicderivativesandintegrals/derivativesproductrule}
    \caption{The geometrical interpretation of the product rule.}
    \label{fig:derivativesproductrule}
\end{figure}
The derivative of $f(x)g(x)$ w.r.t. $x$ can be thought of a rectangle with side length that's governed by $f(x)$ and $g(x)$. As shown in \cref{fig:derivativesproductrule}, the area increase on side $f(x)$ is $g(x)\Dd{f}$ and on $g(x)$, $f(x)\Dd{g}$. The $\Dd{f}\Dd{g}$ part is basically negligible. Therefore,
\begin{align*}
    \odv*{(f(x)g(x))}{x} &= \lim_{h \appr 0}\frac{g(x)\Dd{f} + f(x)\Dd{g}}{h} \\
    &= \lim_{h \appr 0}\frac{f(x)(g(x + h) - g(x)) + g(x)(f(x + h) - f(x))}{h} \\
    &= f(x)\lim_{h \appr 0}\frac{g(x + h) - g(x)}{h} + g(x)\lim_{h \appr 0}\frac{f(x + h) - f(x)}{h} \\
    &= f(x)\odv{g(x)}{x} + g(x)\odv{f(x)}{x}.
\end{align*}
Which is what we call the product rule. Notice the alternation between $f$ and $g$ in the two terms. The derivative of $f(x)$ is multiplied by $g(x)$, and the derivative of $g(x)$ is multiplied by $f(x)$. This is a direct consequence of the diagram: the change in $f(x)$ is multiplied by $g(x)$ to give the area and also the other way around. You could check this with the method of increments, and it would still be true. I encourage you to do it.

\index{derivatives!quotient rule}To take derivatives of quotients of functions, just plug in $\flatfrac{1}{g(x)}$ instead of $g(x)$. The final form should be
\begin{equation}
    \odv*{\ab(\frac{f(x)}{g(x)})}{x} = \frac{\displaystyle f(x)\odv{g(x)}{x} - g(x)\odv{f(x)}{x}}{f(x)^2}.
\end{equation}
But how's about the product of multiple functions? We can't really geometrically interpret those anymore. So we have to turn ourselves to the analytical method.

Let's think of this through. We don't know the derivative of products, but we know the derivative of \emph{sums}: the linearity property. So if we can turn products into sum, this problem would be so easy! Gladly, the logarithms can do exactly that. To make our life easier, we shall use the natural logarithm. Because I want to save space, let's write $a(x)b(x)c(x)$ as just $abc$. Do know that these functions are all dependent on $x$. Start off with a manipulation of products.
\begin{align*}
    \odv{(abc\dots)}{x} &= \odv*{(\e^{\ln(abc\dots)})}{x} \\
    &= \odv*{(\e^{\ln(a) + \ln(b) + \ln(c) + \dots})}{x}.
\end{align*}
Now, let $\ln(a) + \ln(b) + \ln(c) + \dots = u$ and use the chain rule,
\begin{align*}
    &= \odv*{(\e^u)}{x} \cdot \odv{u}{u} \\
    &= \odv*{(\e^u)}{x} \cdot \odv*{\ab(\ln(a) + \ln(b) + \ln(c) + \dots)}{x} \\
    &= \e^u \left(\odv{\ln(a)}{x} + \odv{\ln(b)}{x} + \odv{\ln(c)}{x} + \dots\right).
\end{align*}
Then use the chain rule again on the terms in the parenthesis
\begin{align*}
    &= \e^u \left(\odv{\ln(a)}{x}\odv{a}{a} + \odv{\ln(b)}{x}\odv{b}{b} + \odv{\ln(c)}{x}\odv{c}{c} + \dots\right) \\
    &= (abc\dots)\left(\odv{\ln(a)}{a}\odv{a}{x} + \odv{\ln(b)}{b}\odv{b}{x} + \odv{\ln(c)}{c}\odv{c}{x} + \dots\right) \\
    &= (abc\dots)\left(\frac{1}{a}\odv{a}{x} + \frac{1}{b}\odv{b}{x} + \frac{1}{c}\odv{c}{x} + \dots\right).
\end{align*}
And there we have it: the generalized product rule.

\subsection{Alternative derivations for the power rule}

The power rule can also be derived using the same technique we just used. However, we use a different property of logarithm: $\ln(x^n) = n\ln(x)$.
\begin{equation*}
    \odv*{x^n}{x} = \odv*{\e^{n\ln(x)}}{x}.
\end{equation*}
Let $u = n\ln(x)$ then use the chain rule
\begin{align*}
    &= \odv{\e^u}{x} \cdot \odv{u}{u} \\
    &= \odv{\e^u}{u} \cdot \odv{u}{x} \\
    &= \e^u \cdot \odv*{\ab(n\ln(x))}{x} \\
    &= x^n \cdot n\frac{1}{x} = nx^{n - 1}.
\end{align*}

\section{Implicit differentiation}

Implicit differentiation also shows up in problems where two rate of changes are related to each other: the \textbf{related rates} problem. Consider a sliding ladder that's $5\unit{\meter}$ long (\cref{fig:slidingladder}). At a certain time $t$, what's the sliding rate along the $x$-axis w.r.t. the $y$-axis?\index{related rates}

\begin{figure}
    \centering
    \includegraphics{basicderivativesandintegrals/slidingladder}
    \caption{A ladder length $5\unit{\meter}$ sliding down a corner.}
    \label{fig:slidingladder}
\end{figure}

The problem is asking for $\odv{x}{y}$. Here, the rate of sliding along the $y$-axis $\odv{y}{t}$ and along the $x$-axis $\odv{x}{t}$ is related because the ladder length is fixed. If $y$ decreases, $x$ must increase. The Pythagorean theorem says
\begin{equation}
    x^2 + y^2 = 5^2.
\end{equation}
We can take the derivative w.r.t. $t$ on both side then use the chain rule
\begin{align*}
    \odv{x^2}{t} + \odv{y^2}{t} &= \odv{(5^2)}{t} \\
    \odv{x^2}{x}\odv{x}{t} + \odv{y^2}{y}\odv{y}{t} &= 0 \\
    2x\odv{x}{t} + 2y\odv{y}{t} &= 0.
\end{align*}
We can take advantage of the Leibniz's notation and multiply by $\odif{t}$ on both sides giving
\begin{equation*}
    2x\odif{x} + 2y\odif{y} = 0.
\end{equation*}
To find $\odv{x}{y}$, we just have to isolate the variables,
\begin{align*}
    \odv{x}{y} = -\frac{y}{x}.
\end{align*}
And this should actually make sense. Because if $y$ is increasing by a bit, $x$ must decrease by some amount, and that amount is $-\flatfrac{y}{x}$: the higher the $y$, the larger the rates of sliding.

\section{But why is the integral of the reciprocal the natural logarithm?}
\label{sec:integralofthereciprocal}

\index{integrals!integrals of the reciprocal}
\begin{wrapfigure}{r}{0.45\textwidth}
    \centering
    \begin{tabular}{c | c}
        Function & Antiderivative \\
        \hline
        $x^{-3}$ & $-\flatfrac{x^{-2}}{2}$ \\
        $x^{-2}$ & $x^{-1}$ \\
        $x^{-1}$ or $\flatfrac{1}{x}$ & $\ln(x)$ \\
        $x^{0}$ or $1$ & $x$ \\
        $x^1$ & $\flatfrac{x^2}{2}$ \\
    \end{tabular}
    \caption{Tables of reversed power rule from $x^{-3}$ to $x^1$}
\end{wrapfigure}
By the fundamental theorem of calculus (\cref{theorem:fundamentalthmofcalc1}), the antiderivative of $\frac{1}{x}$ is $\ln(x)$. You might believe this, and it's not wrong. But I find it very disturbing and unresolved. It's like taking the result and pointing it back to the origin. It doesn't really make sense. So from this dissatisfaction, I spent a night coming up with a way to derive this using just the reversed power rule. Enjoy the transformation!
\begin{align}
	\int \frac{1}{x}\odif{x} &= \int\lim_{h \appr 0}\left(\frac{1}{2}x^{-1 + h} + \frac{1}{2}x^{-1 - h}\right)\odif{x} \nonumber\\
							 &= \lim_{h \appr 0}\int\left(\frac{1}{2}x^{-1 + h} + \frac{1}{2}x^{-1 - h}\right)\odif{x} \nonumber\\
							 &= \lim_{h \appr 0}\int\left(\frac{1}{2}\frac{x^{-1 + h + 1}}{(-1 + h + 1)} + \frac{1}{2}\frac{x^{-1 - h + 1}}{(-1 - h + 1)}\right)\odif{x} \nonumber\\
    &= \lim_{h \appr 0}\left(\frac{1}{2}\frac{x^h}{h} - \frac{1}{2}\frac{x^{-h}}{-h}\right) = \lim_{h \appr 0}\left(\frac{1}{2}\frac{x^h}{h}\frac{x^h}{h} - \frac{1}{2}\frac{1}{hx^h}\right) \nonumber\\
    &= \lim_{h \appr 0}\left(\frac{x^{2h} - 1}{2hx^h}\right) = \lim_{h \appr 0}\left(\frac{\e^{2h\ln(x)} - 1}{2h\ln(x)}\cdot\frac{\ln(x)}{x^h}\right) \nonumber\\
    &= \lim_{h \appr 0}\left(\frac{\e^{2h\ln(x)}}{2h\ln(x)}\right) \cdot \lim_{h \appr 0}\left(\frac{\ln(x)}{x^h}\right) \nonumber\\
    &= \lim_{h \appr 0}\left(\frac{\e^{2h\ln(x)}}{2h\ln(x)}\right) \cdot \ln(x). \label{eq:logarithmsprove1}
\end{align}
Then, we evaluate the limit at the front by letting $u = 2h\ln(x)$. When $h \appr 0$, $u \appr 0$ as well. Then, use the definition of $\e$ from \cref{eq:infinitecompoundinterest}.
\begin{align*}
    \lim_{h \appr 0}\left(\frac{\e^{2h\ln(x)}}{2h\ln(x)}\right) &= \lim_{u \appr 0}\left(\frac{\e^u - 1}{u}\right) \\
    &= \lim_{u \appr 0}\left(\frac{\left(\lim_{n \appr \infty}\left(1 + \frac{1}{n}\right)^n\right)^u}{u}\right).
\end{align*}
Change the limits from $n \appr \infty$ into $n \appr 0$. Notice, $\lim_{n \appr 0}\left(1 + \frac{1}{n}\right)^n = \lim_{n \appr 0}\left(1 + n\right)^{\flatfrac{1}{n}}$. If $n \appr 0$ and $u \appr 0$, that means $n = u$. Substitute $n = u$ into the limit,
\begin{align*}
    &= \lim_{u \appr 0}\left(\frac{\left(\lim_{u \appr 0}(1 + u)^{\flatfrac{1}{u}}\right)^u - 1}{u}\right) \\
    &= \lim_{u \appr 0}\left(\frac{1 + u - 1}{u}\right) = 1.
\end{align*}
Then, substitute this limit back into \cref{eq:logarithmsprove1}, you'll see that
\begin{equation*}
    \int\frac{1}{x}\odif{x} = \lim_{h \appr 0}\left(\frac{\e^{2h\ln(x)}}{2h\ln(x)}\right) \cdot \ln(x) = \ln(x) + C,
\end{equation*}
completing the proof.

\section{Formula for Chapter \thechapter}

\everymath{\displaystyle}
\subsection{Formula for derivatives of functions}

\begin{enumerate}
    \item $f(x) = g(x) \implies \odv{f(x)}{x} = \odv{g(x)}{x}$ (Rules of Equality)
    \item For $c \in \mathbb{R}$, $\odv{(c)}{x} = 0$ (Derivative of a constant)
    \item $\odv{x}{y} = \odv{x}{u} \times \odv{u}{y}$ (Chain rule)
    \item $\odv*{\ab(af(x) + bg(x))}{x} = a\odv{f(x)}{x} + b\odv{g(x)}{x}$ (Linearity of differentiation)
    \item $\odv*{(ax^n)}{x} = anx^{n - 1}$ (Power rule)
    \item $\odv*{(n^x)}{x} = n^x\ln(n)$, $\odv{(\e^x)}{x} = \e^x$ (Derivative of exponentials)
    \item $\odv*{(\ln(x))}{x} = \frac{1}{x}$ (Derivative of natural logarithms)
	\item $\odv*{\ab(f_1(x)f_2(x))}{x} = f_1(x)\odv{f_2}{x} + f_2(x)\odv{f_1}{x}$ (Product rule for two functions)
	\item $\odv*{(f_1f_2\dots f_n)}{x} = f_1f_2\dots f_n\ab(\frac{1}{f_1}\odv{f_1}{x} + \frac{1}{f_2}\odv{f_2}{x} + \dots + \frac{1}{f_n}\odv{f_n}{x})$ (Generalized product rule)
\end{enumerate}

\subsection{Formula for antiderivatives of functions}

\begin{enumerate}
    \item $f(x) = g(x) \implies \int f(x)\odif{x} = \int g(x)\odif{x}$ (Rules of Equality)
    \item $\int af(x) + bg(x) \odif{x} = a\int f(x)\odif{x} + b\int g(x)\odif{x}$ (Linearity of integration)
    \item $n \neq -1$, $\int ax^n\odif{x} = a\frac{x^{n + 1}}{n + 1} + C$ (Reversed power rule)
    \item $\int n^x\odif{x} = \frac{1}{\ln(n)}n^x + C$, $\int \e^x\odif{x} = \e^x + C$ (Antiderative of exponentials)
    \item $\int \frac{1}{x}\odif{x} = \ln(x) + C$ (Antiderivative of natural logarithms)
\end{enumerate}

\subsection{Definition for various functions and constants}

\begin{enumerate}
	\item $\e^x = \lim_{k \appr 0}\sum_{i \appr 0}^{k}\frac{1}{i!} = 1 + x + \frac{x^2}{2!} + \frac{x^3}{3!} + \dots$
	\item $\e = \lim_{n \appr \infty}\left(1 + \frac{1}{n}\right)^n$, and $\e = \sum_{i = 0}^{\infty}\frac{1}{i!} = 1 + \frac{1}{1!} + \frac{1}{2!} + \frac{1}{3!} + \dots$
    \item $\ln(x) = \lim_{n \appr 0}\left(\frac{x^h - 1}{h}\right)$
\end{enumerate}

\everymath{\textstyle}

\chapter{A study of graphs with calculus}

\section{Invitation: sketching mountain ranges}

If I told you to \emph{sketch} out $x^2$, it'd be easy, right? Just a simple parabola would do. But if I say: plot $x^4 - 2x^3 + x^2 - x + 1$? It isn't so easy now is it. You could substitute various $x$ and plot it into the graph directly. But I'm asking for a rough sketch, not a plot. I just want the general shape and structure of the function: where are the highest and lowest points, the $x$-intercept, and the $y$-intercept. We're going to learn how to do exactly that in this chapter.

Imagine a function as a vast mountain range. When sketching this landscape, we're interested in identifying its peaks, valleys, and plateaus. A peak represents the sharp summit of a mountain, where the land falls away in all directions; a valley is a sunken area, nestled lower than the surrounding terrain; and a plateau is a level stretch, a temporary calm before the ascent resumes toward higher peaks or descends into deeper valleys. Standing at a peak, the land around you dips downward. In contrast, in a valley, everything rises above you. On a plateau, the ground is flat under your feet, but when gazed into the distance, one side raises higher, but the other reveals the path down. These important features are all characterized by a flat ground: a place with zero slope. If the mountain range represents function, then all of these points have zero first derivative.

Still, the information about zero first derivative is still not enough to determine whether that point is a peak, a valley, or a plateau. The information able to classify these points are the sign of the second derivative.

The second derivative suggest the concavity of a graph at that point. If the first derivative gives you the slope, the second derivative will give you the rate of change of that slope. If the second derivative is positive, it means that the function is increasing around that point; thus, that point must be a valley. If it's negative, then everything is decreasing around; thus, it's a peak. If it's zero, it's simply a plateau.

Just like a mountain, a function can have many points that has zero derivatives. Functions with more than one peaks exists. The first derivative does not provide us with any information on which one of those peaks are the highest. It only tells you where the peak is. In order to find which one is the highest, we'd just have to directly find the function's value at those points, then compare it.

In mathematics terminology, the points that have zero first derivative is called a \textbf{stationary point}: the value of the function around it is stationary. The valleys are called, \textbf{local minima}; the peaks, \textbf{local maxima}, and the plateau, \textbf{inflection point}. The word \emph{local} suggests that locally, these are either the maximum or minimum point, but it might not be globally. Like how you can stand on your house and say that \enquote{this is the highest point of my house,} but you're still not higher than mount everest. We call the global maximum or minimum point, the \textbf{absolute maximum} and \textbf{absolute minimum}. \index{local minima}\index{local maxima}\index{absolute maximum}\index{absolute minimum}

\section{Minima and maxima}

This section focuses on finding the minima and maxima of functions and sketching it. These methods are applicable for all kinds of function, but we'll only study the important ones: quadratics and cubics.

\subsection{Shape of quadratics}

Let's start off with quadratics: a function with only one peak or valley. For any given quadratic
\begin{equation}
	f(x) = ax^2 + bx + c,
\end{equation}
Its derivative can be found using the power rule (\cref{eq:power_rule}).
\begin{align}
	\odv*{f(x)}{x} &= \odv*{\ab(ax^2 + bx + c)}{x} \\
				   &= 2ax + b.
\end{align}
The peak or valley of $f(x)$ is at $x = x_0$ where $\odv*{f(x)}{x}_{x = x_0} = 0$. The notation $\odv*{f(x)}{x}_{x = x_0}$ tells you that this derivative needs to be evaluated at $x = x_0$. For a quadratic, this means
\begin{align}
	2ax_0 + b &= 0 \\
	x_0 &= -\frac{b}{2a},
\end{align}
which matches with the usual result from traditional algebra. The $y$ value of these points can be found by plugging in $x = -\frac{b}{2a}$ into $f(x)$.
\begin{align}
	f\ab(-\frac{b}{2a}) &= a\ab(-\frac{b}{2a})^2 + b\ab(-\frac{b}{2a}) + c \\
						&= \frac{b^2}{4a} - \frac{b^2}{2a} + c \\
						&= c - \frac{3b^2}{4a} \label{eq:parabola_y_vertex}
\end{align}

Other important points include the $y$ and $x$ intercept. The $y$-intercept can easily be found by setting $x = 0$:
\begin{equation}
	y_0 = a(0)^2 + b(0) + c \\
	y_0 = c.
\end{equation}
Sadly, calculus does not offer any tools to find the exact root of an equation; you still have to use the quadratic equation
\begin{equation}
	x = \frac{- b \pm \sqrt{b^2 - 4ac}}{2a}.
\end{equation}

Classically, the direction that the quadratic The second derivative of a quadratic is determined by the sign of $a$. The same result can be derived from the second derivative.
\begin{align}
	\odv*[ord = 2]{f(x)}{x} &= \odv*{\ab(\odv{f(x)}{x})}{x} \\
							&= \odv*{\ab(2ax + b)}{x} \\
							&= 2a.
\end{align}
Since we only care about the sign, the two does not matter. This results tell us that if $a$ is positive, it's a right side up parabola, and the stationary point is a local minima. If it isn't, then it's upsidedown and the stationary point is a local maxima.

A parabola doesn't have any inflection point because it only occurs when the second derivative is zero. However, the only way that $\odv*[ord = 2]{f(x)}{x} = 0$ is for $a$ to be zero. If it happens, then $f(x)$ isn't a parabola anymore, but rather a straight line.

\begin{exmp}{Sketch the graph of $f(x) = - 2x^2 + x + 6$}{}
	Because $a$ is negative, the parabola is upsidedown. The vertex is at
	\begin{gather}
		x = -\frac{b}{2a} = -\frac{1}{2(-2)} = \frac{1}{4}, \\
		\quad y = -2\ab(\frac{1}{4})^2 + \ab(\frac{1}{4}) + 6 = 6.125.
	\end{gather}
	The $y$-intercept is trivially $6$. The $x$ intercept are the roots of this polynomial, given by
	\begin{equation}
		x = \frac{-(1) \pm \sqrt{(1)^2 - 4(-2)(6)}}{2(-2)} = 2, -\frac{3}{2}.
	\end{equation}
	Factorization would also yield the same result.
	\begin{align}
		-2x^2 + x + 6 &= -(2x^2 - x - 6) \\
					  &= -(x - 2)(2x + 3) \rightarrow x = 2, -\frac{3}{2}
	\end{align}
	Therefore, the sketched graph would be
	\begin{figure}[H]
		\centering
		\includegraphics[height = 4cm]{geometryandcalculus/parabolasketch.pdf}
		\caption{Sketch of the function $-2x^2 + x + 6$.}
	\end{figure}
\end{exmp}

We can also determine the amount of roots by just looking at the $y$-value of the vertex point, given by \cref{eq:parabola_y_vertex}, and the direction of the parabola. But it's more computationally expensive than the determinant method which says,
\begin{displayquote}
	A parabola has no roots if $b^2 - 4ac < 0$, one root if $b^2 - 4ac = 0$, and two roots if $b^2 - 4ac > 0$.
\end{displayquote}
This is a direct result of the quadratic formula: $x = \frac{- b \pm \sqrt{b^2 - 4ac}}{2a}$ only have real solution if the terms in the root, $b^2 - 4ac$, is positive.

\subsection{Shape of cubics}

Cubics are third order polynomials
\begin{equation}
	f(x) = ax^3 + bx^2 + cx + d.
\end{equation}
It's one of the many family of function that can contain more than one stationary point. The $x$ coordinate of the stationary points can be found by using the same method as earlier.
\begin{align}
	\odv{f(x)}{x} &= \odv*{ax^3 + bx^2 + cx + d}{x} \\
				  &= 3ax^2 + 2bx + c,
\end{align}
in which
\begin{align}
	\odv{f(x)}{x}_{x = x_0} = 0 &= 3ax_0^2 + 2bx_0 + c \\
	x_0 &= \frac{- 2b \pm \sqrt{(2b)^2 - 4(3a)(c)}}{2(3a)} \\
		&= \frac{- 2b \pm 2\sqrt{b^2 - 3ac}}{6a} \\
		&= \frac{- b \pm \sqrt{b^2 - 3ac}}{3a}. \label{eq:cubic_vertex_position-x}
\end{align}
By the terms under the root, a cubic has no stationary point if $\frac{b^2 - 3ac}{3a} < 0$, one if $\frac{b^2 - 3ac}{3a} = 0$, and two if $\frac{b^2 - 3ac}{3a} > 0$.

The function $f(x)$ intersects the $y$ axis when $x = 0$:
\begin{equation}
	f(0) = a(0)^3 + b(0)^2 + c(0) + d = d;
\end{equation}
thus, the $y$ intercept is at $(0, d)$. For the $x$-intercept, again, there is no tool in calculus that can be used to find the exact root. So, you'd just have to either factor the function, or use the very lengthy cubic formula.
\begin{equation}
	\begin{gathered}
		x = \sqrt[3]{q + \sqrt{q^2 + (r - p^2)^3}} + \sqrt[3]{q - \sqrt{q^2 + (r - p^2)^3}} + p
	\end{gathered}
\end{equation}
where
\begin{equation}
	p = -\frac{b}{3a}, \quad q = p^3 + \frac{bc - 3ad}{6a^2} \mathand r = \frac{c}{3a}.
\end{equation}

% Insert examples here

Another topic that I'd like to discuss is the amount of root of cubics. Normally, we would use factorization to determine the amount. From the shape of the cubic, one side shoots of to positive infinity, and the other, negative infinity. And by continuity, it must intersect the $x$ axis somewhere inbetween and create a root. Therefore, all cubics must have at least one root over the reals. But what if we want to know whether it has a second or third roots or not? We can just look at the amount and the position of the stationary point, given by \cref{eq:cubic_vertex_position-x}.

A key feature of stationary points is that they indicate where a function changes direction. If a function is increasing and then begins to decrease, there must be a stationary point marking this transition. So, if a function lacks a stationary point, and one end tends toward positive infinity while the other falls to negative infinity, we can conclude the function is always either increasing or decreasing. Once the function intersects the $x$-axis to form a root, it cannot turn back to create another root.

A cubic with zero stationary points ($\frac{b^2 - 3ac}{3a} < 0$) are such kind of function. One side goes to infinity, and the other, negative infinity. Since it has no stationary point, it must only have one root. One such example is the function $f(x) = x^3 + x^2 + x$, plotted in \cref{fig:cubic_no_vertex}.

\begin{figure}[t]
	\centering
	\begin{subfigure}[t]{0.3\textwidth}
		\centering
		\includegraphics{geometryandcalculus/cubic_novertex.pdf}
		\caption{$x^3 + x^2 + x$: No vertex}
		\label{fig:cubic_no_vertex}
	\end{subfigure}
	\begin{subfigure}[t]{0.3\textwidth}
		\centering
		\includegraphics{geometryandcalculus/cubic_onevertex.pdf}
		\caption{$x^3$: One vertex}
		\label{fig:cubic_one_vertex}
	\end{subfigure}
	\caption{Cubics with zero and one vertex}
\end{figure}

Similarly, cubics with only one stationary point ($\frac{b^2 - 3ac}{3a} = 0$) only have one root. Illustrated in \cref{fig:cubic_one_vertex}, the stationary point must be an inflection point, as it cannot represent a local maximum or minimum. This is due to the behavior of the function, which extends toward both positive and negative infinity. If the stationary point were a local maximum or minimum, it would turn into a parabola, and thus needs another point to reverse its direction again to be a cubic. Hence, cubic functions with only one stationary point have just one root.

\begin{figure}[ht]
	\centering
	\begin{subfigure}[t]{0.3\textwidth}
		\centering
		\includegraphics[]{geometryandcalculus/cubic_twovertex_threeroots.pdf}
		\caption{$x^3 + x$: Three roots}
		\label{fig:cubic_two_vertices_threeroots}
	\end{subfigure}
	\begin{subfigure}[t]{0.3\textwidth}
		\centering
		\includegraphics[]{geometryandcalculus/cubic_twovertex_tworoots.pdf}
		\caption{$x^3 - \frac{3}{4}x + \frac{1}{4}$: Two roots}
		\label{fig:cubic_two_vertices_tworoots}
	\end{subfigure}
	\begin{subfigure}[t]{0.3\textwidth}
		\centering
		\includegraphics[]{geometryandcalculus/cubic_twovertex_oneroot.pdf}
		\caption{$x^3 - x$: One root}
		\label{fig:cubic_two_vertices_oneroot}
	\end{subfigure}
	\caption{Two vertices cubics with varying amount of roots}
	\label{fig:cubic_two_vertices}
\end{figure}

For cubics with two vertex, one of those must be a local maximum, the other, a local minimum. The amount of roots can be directly determined by the position of these points. A cubic will have three roots if the vertex aren't on the same side of the $x$ axis, it must have three roots, illustrated in \cref{fig:cubic_two_vertices_threeroots}. If one of the vertex is at $y = 0$, then it has two roots; one of them being the vertex itself, illustrated in \cref{fig:cubic_two_vertices_tworoots}. If both vertices are on the same side, then it must have only one root; illustrated in \cref{fig:cubic_two_vertices_oneroot}. It is left as an exercise for the reader to find the position of vertices of the cubics given in \cref{fig:cubic_two_vertices}

\subsection{Optimization problems}

The method of finding local maxima, minima, and inflection points are very useful for sketching graphs. Are there any real life applications to this? The answer is in optimization.

\textbf{Optimization} is about finding the best solution for a problem. What's the best time to harvest the crop? What's the best placement for an air conditioner? What's the best way to make profit?, etc. These problems are all about finding the maximum values of some variables. That's exactly what our tools have been able to do: finding the maxima and minima of some functions. Thus, I'd like to discuss about some applications of maxima and minima on some physical problems. \index{optimization problems}

\begin{wrapfigure}{r}{0.3\textwidth}
	\centering
	\includegraphics{geometryandcalculus/optimization_rectangle_area_fixed_perimeter.pdf}
	\caption{A rectangle for optimization}
	\label{fig:optimization_rectangle_area_fixed_perimeter}
\end{wrapfigure}
\paragraph{Classic maximum area problem.} Given a rope length $L$, what's the maximum rectangular area that this rope can enclose?

To solve this problem, we start by constructing our rectangle. Drawn in \cref{fig:optimization_rectangle_area_fixed_perimeter}, the perimeter of a rectangle is given by $2(a + b) = L$. Without loss of generality, let side $a$ has length $l$. Thus,
\begin{align}
	2(l + b) &= L \\
	b &= \frac{L}{2} - l
\end{align}
The area of the rectangle, which is a function of $l$, $A(l)$ is then
\begin{equation}
	A(l) = a \times b = l\ab(\frac{L}{2} - l) = -l^2 + l\frac{L}{2}.
\end{equation}
Since this function is a parabola, its stationary point must be the local maxima, and must also be the global maxima. This point is where our function is the highest. It can be found by using the derivative w.r.t. $l$.
\begin{equation}
	\odv*{A(l)}{l} = -2l + \frac{L}{2}
\end{equation}
Using the same argument as earlier, we want to find the point $l_0$ where $\odv*{A(l)}{l}_{l = l_0} = 0$:
\begin{align}
	0 &= - 2l_0 + \frac{L}{2} \\
	l_0 &= \frac{L}{4};
\end{align}
thus, the maximum $A(l)$ is
\begin{equation}
	A\ab(\frac{L}{4}) = -\ab(\frac{L}{4})^2 + \ab(\frac{L}{4})\frac{L}{2} = \frac{L^2}{16}
\end{equation}
Funnily enough, the maximum area that a perimeter length $L$ can increase is also a parabola $\frac{1}{16}L^2$.

\paragraph{Modified maximum area problem.} Given a fence length $L$. What's the maximum area that I can enclose, given that one of the sides doesn't need any fence?

If our rectangle has sides $a$ and $b$, and one side doesn't need a fence, without loss of generality, the perimeter is $a + 2b = L$. If $a = l$, then $b = \frac{L - l}{2}$. The area of the rectangle then becomes
\begin{equation}
	A(l) = a \times b = l\ab(\frac{L - l}{2}) = - \frac{1}{2}l^2 + \frac{L}{2}l. 
\end{equation}
Using the same argument,
\begin{align}
	\odv*{A(l)}{l} &= -l + \frac{L}{2} \\
	\odv*{A(l)}{l}_{l = l_0} = 0 &= -l_0 + \frac{L}{2} \\
	l_0 &= \frac{L}{2},
\end{align}
and
\begin{align}
	A(l_0) = -\frac{1}{2}\ab(\frac{L}{2})^2 + \frac{L}{2}\ab(\frac{L}{2}) = \frac{3}{8}L^2.
\end{align}

\paragraph{Volume of a box.} Let's say you have a paper size $a \times b$, shown in \cref{fig:optimization_box_volume_unfold}. If you want to fold it up into a box without a closing top by cutting a square size $x$ on each corner, shown in \cref{fig:optimization_box_volume_folded}, find the maximum volume that the box can enclose.

\begin{figure}[ht]
	\centering
	\begin{subfigure}[b]{0.51\textwidth}
		\centering
		\includegraphics[]{geometryandcalculus/optimization_box_volume_unfold.pdf}
		\caption{Unfolded. The gray part is cut out.}
		\label{fig:optimization_box_volume_unfold}
	\end{subfigure}
	\begin{subfigure}[b]{0.43\textwidth}
		\centering
		\includegraphics[]{geometryandcalculus/optimization_box_volume_folded.pdf}
		\caption{Folded}
		\label{fig:optimization_box_volume_folded}
	\end{subfigure}
	\caption{Box volume optimization}
\end{figure}

Shown in \cref{fig:optimization_box_volume_unfold}, if we cut out a square sidelength $x$ from all corners, then the sidelength of the final box will be $a - 2x$, $b - 2x$, and $x$. The volume $V(x)$, is then
\begin{equation}
	V(x) = x(a - 2x)(b - 2x) = 4x^3 + (- 2a - 2b)x^2 + (ab)x
\end{equation}
This function is a cubic, and we can use \cref{eq:cubic_vertex_position-x} to find the $x$ coordinate of the vertex.
\begin{align}
	x_0 &= \frac{- (- 2a - 2b) \pm \sqrt{(- 2a - 2b)^2 - 3(4)(ab)}}{3(4)} \\
		&= \frac{2a + 2b \pm \sqrt{4a^2 + 4b^2 - 4ab}}{12} \\
		&= \frac{a + b \pm \sqrt{a^2 + b^2 - ab}}{6}
\end{align}
If $a^2 + b^2 - ab = 0$, there's only one stationary point at $x_0 = \frac{a + b}{6}$, which is an inflection point. Otherwise, it must have two stationary points. Since the leading coefficient of $V(x)$ is positive, $V(x)$ must approach negative infinity when $x$ approaches infinity. Meaning that, the highest local maxima is on the right vertex; thus, we select the negative root:
\begin{equation}
	x_0 = \frac{a + b - \sqrt{a^2 + b^2 - ab}}{6}.
\end{equation}
The full solution is left out due to arithmetical tediousness. It'd be better if you just plug in the numbers.

\section{Behaviors at infinity: asymptotes and rational functions}

Polynomials are very well behaved, in a sense that if $f(x)$ is a polynomial, the only way that $f(x) \appr \infty\infty$, is that $x \appr \pm\infty$. However, there are other family of functions which can have infinities anywhere that it likes: the \textbf{rational functions}.\index{rational functions}

A rational function $F(x)$ is a function that can be written as a ratio of two polynomials:
\begin{equation}
	F(x) = \frac{P(x)}{Q(x)} = \frac{p_0 + p_1x^1 + \dots + p_nx^n}{q_0 + q_1x^1 + \dots + q_mx^m}.
\end{equation}
By the fundamental theorem of algebra, we can always factor these polynomials into
\begin{equation}
	\begin{multlined}
		P(x) = (x - P_0)(x - P_1)\dots(x - P_n), \\
		Q(x) = (x - Q_0)(x - Q_1)\dots(x - Q_n),
	\end{multlined}
\end{equation}
where $P_0, \dots, P_n, Q_1, \dots, Q_n$ are components of the complex plane; thus,
\begin{equation}
	F(x) = \frac{(x - P_0)(x - P_1)\dots(x - P_n)}{(x - Q_0)(x - Q_1)\dots(x - Q_n)}.
\end{equation}
When $x$ approaches any of the $Q$'s in the denominator, $Q(x) \appr 0$, and thus $F(x) \appr \infty$ there. The point $x = Q_1, Q_2, \dots, Q_n$ are called the \textbf{singularities} of $F(x)$.

For example, the function $F(x) = \frac{1}{x}$, sketched in \cref{fig:reciprocal-function} explodes to $\pm\infty$ at $x \appr 0$, but gradually decreases to $0$ when $x \appr \pm\infty$. To study these functions, we have to define a line called the \textbf{asymptote}.
\begin{figure}[ht]
	\centering
	\includegraphics{geometryandcalculus/reciprocal-function.pdf}
	\caption{Graph of $\frac{1}{x}$}
	\label{fig:reciprocal-function}
\end{figure}

An \textbf{asymptote} is a line that the graph of a function approaches but never intersect. E.g., the function $F(x) = \frac{1}{x}$ has two asymptotes: the line that runs through the $x$ axes, and the line that runs through the $y$-axis. Generally, an asymptote could be horizontal, vertical, or even slanted.

\section{Tangent to a curve}

In this section, I'll show you how to find a line that's tangent to a curve. But to do it the traditional way would be a bit boring, as it's quite dry and actually doesn't help us with graph sketching at all. So, I shall introduce it via the Newton-Raphson's root finding algorithm; as it's an algorithm that directly takes advantage of the line tangent to a function, and it shows up in many places. So much so that it exists inside the core of the fast square root algorithm which is implemented in every computer.

\subsection{Newton-Raphson method}

Polynomials appears everywhere. Most of the time, you're required to find its root. But it might come as a surprise for you that it is \emph{impossible} to find a general for finding the root of polynomials with degrees higher than four. This is a direct consequence of Galois theory\footnote{The Galois theory is a profound result from abstract algebra. You would usually learn in graduate mathematics, but if you want to just scratch the surface, I found a video by \emph{Math Visualized}, on \emph{Galois Theory Explained Simply} \cite{galois-explained-2020}, that explains it quite clearly. Still, I need to watch it a couple of times before understanding it.}, one of the most important theory in abstract algebra. So most of the time, we resort to root approximation methods.

The root approximation method which is relevant to calculus is the Newton-Raphson's root finding algorithm. I'd try to explain this algorithm has much as I can, but I think it's better if you see it.

The Newton-Raphson algorithm takes advantage of slopes descent. Let's say you want to approximate the root of $f(x)$. You put down a random guess, say $x_1$. If that point is a root, $f(x_1) = 0$. If its not, the root is either above ($f(x_1) < 0$) the point or under ($f(x_1) > 0$). What you do next is that you draw a line tangent to the function $f(x)$ at $x = x_1$ and let it cross the $x$-axis, shown in \cref{fig:newton-raphson-method}. Intuitively, the tangent line will intersect the $x$-axis at $x_2$ where $x_2$ should hopefully be closer to the root. Our new guess will then be $(x_2, f(x_2))$. We then repeat this method until the approximation are satisfactory.

Here, I shall use the Newton-Raphson root finding algorithm as a way to introduce you about how calculus can be further applied to geometry.

\begin{figure}[ht]
	\centering
	\includegraphics{geometryandcalculus/newton-raphson-method.pdf}
	\caption{Newton-Raphson's method illustrated}
	\label{fig:newton-raphson-method}
\end{figure}

The first part of the Newton-Raphson method is to be able to find the tangent to a curve. Let's first start with a line equation
\begin{equation}
	y = mx + c.
\end{equation}
If my first guess is on the point $(x_1, f(x_1))$, the line that passes through that point and tangent to $f(x)$ is
\begin{equation}
	y = \odv{f(x)}{x}_{x = x_1}x + c.
\end{equation}
where $m = \odv{f(x)}{x}_{x = x_1}$. And $c$ is the intersection with the $x$ axis.

But we don't have to actually find $c$. For a line tangent to a curve, we just want to move the line from the origin to the tangent point $(x_1, f(x_1))$. To move the line to the right by $x_1$ units, we substitute $x$ with $x - x_1$, and to shift it up by $y_1$, substitute $y$ with $y - y_1$. Thus, the line equation turns into
\begin{align}
	y - y_1 &= (x - x_1)\odv{f(x)}{x}_{x = x_1} \\
	y - f(x_1) &= (x - x_1)\odv{f(x)}{x}_{x = x_1}.
\end{align}
For simplicity of notation, I shall let $\mdif{x}f(x_1)$ represent $\odv{f(x)}{x}$:
\begin{equation}
	y - f(x_1) = (x - x_1)\mdif{x}f(x_1).
\end{equation}

Then, Newton-Raphson method tells us that the intersection between the tangent line and the $x$ axis is what our next guess, $x_2$, should be. It can be easily evaluated by setting $y = 0$.
\begin{align}
	0 - f(x_1) &= (x_2 - x_1)\mdif{x}f(x_1) \\
	x_2 &= x_1 - \frac{f(x_1)}{\mdif{x}f(x_1)}
\end{align}
Generally, we can repeat this forever. Thus,
\begin{equation}
	x_{n + 1} = x_n - \frac{f(x_n)}{\mdif{x}f(x_n)} \label{eq:newton-raphson-method}
\end{equation}
a recurrence relation for finding roots of functions. Notice that we don't even need to know the $y$ value for this algorithm to work. And that's the beauty of it.

Let's use this root finding method to approximate $\sqrt{2}$, which is the root of the polynomial $x^2 - 2 = 0$. Since the derivative of $x^2 - 2$ is $2x$, Newton-Raphson method (\cref{eq:newton-raphson-method}) gives
\begin{align}
	x_{n + 1} &= x_n - \frac{x_n^2 - 2}{2x_n} \\
			  &= \frac{x_n^2 + 2}{2x_n}.
\end{align}
Since $\sqrt{2} \approx 1.414213\dots$, let's use $x_1 = 1$ as our first guess.
\begin{align}
	x_2 &= \frac{x_1^2 + 2}{2x_1} \\
		&= \frac{1^2 + 2}{2(1)} = 1.5,
\end{align}
which is closer to $1.41$ than our first guess. Next,
\begin{align}
	x_3 &= \frac{x_2^2 + 2}{2x_2} \\
		&= \frac{1.5^2 + 2}{2(1.5)} = 1.4166\dots,
\end{align}
which is off by just a $0.2\%$ error. By the forth iteration, we get $1.414215\dots$, which is off by a mere $0.0002\%$. And we can stop here.

\subsection{Numerical version of the Newton-Raphson algorithm}

For a function that doesn't have a derivative, or you're just too lazy to explicitly find the derivative, you can turn the Newton-Raphson method into a purely numerical algorithm. By using the method of increments in \cref{eq:newton-raphson-method},
\begin{align}
	x_{n + 1} &= x_n - \frac{f(x_n)}{\lim_{h \appr 0}\frac{f(x_n + h) - f(x_n)}{h}} \\
			  &= x_n - \lim_{h \appr 0}\frac{f(x_n)h}{f(x_n + h) - f(x_n)}.
\end{align}
Computationally, we can set $h$ to any arbitrarily small values: $0.01$ for example. If you have some experiences with coding, this can be easily implemented by using a \texttt{for} loop. Here's some python code that I've written to do just that. Just put in the function in \texttt{def f(x):} block, put the amount of iteration in \texttt{iterationCount}, your first guess in \texttt{firstGuess}, and the value of $h$ in \texttt{derivativeStep}. If you have some experiences, I highly encourage you to tinker around.

\begin{minted}{python}
import matplotlib.pyplot as plt
import math


def f(x):  # Insert function here
    return x**3 - 2 * x + 1

iterationCount = 10  # Amount of iteration
firstGuess = -2  # First guess
derivativeStep = 0.01  # For calculating the method of increments
approximationList = [float(firstGuess)]  # List of approximations


for i in range(iterationCount):
    diff = (
        f(approximationList[i] + derivativeStep) - f(approximationList[i])
    ) / derivativeStep  # Finding the derivative at a certain point
    nextGuess = (
        approximationList[i] - f(approximationList[i]) / diff
    )  # Newton-Raphson's method
    approximationList.append(nextGuess)  # Add next guess to the list of approximations

print(approximationList)  # Print out the list of approximations
fig, axes = plt.subplots()  # Initiate plot
axes.plot(list(range(iterationCount + 1)), approximationList)  # Plotting
axes.set_xlabel("Iteration")  # Setting the x-axis label
axes.set_ylabel("Guess")  # Setting the y-axis label
plt.show()  # Showing the plot
\end{minted}

\section{Normal lines}
\label{sec:normal-lines}

Alongside the tangent line, there is another line that we're interested in: the normal line, a line that's perpendicular to the tangent. It's actually pretty simple to solve. By analytical geometry, if two lines are perpendicular to each other, the product of their slope must be $-1$.

If the slope of tangent is $m_T$ and the normal is $m_N$,
\begin{align}
	m_T \times m_N &= -1 \\
	m_N &= -\frac{1}{m_T}.
\end{align}
Since the slope of the tangent at a point $(x_0, f(x_0))$ is $\mdif{x}f(x_0)$,
\begin{equation}
	m_N = -\frac{1}{\mdif{x}f(x_0)}.
\end{equation}
The normal line equation can then be formed by translating the line with slope $m_N$ to the point $(x_0, f(x_0))$:
\begin{equation}
	y - f(x_0) = -\frac{x - x_0}{\mdif{x}f(x_0)}.
\end{equation}

\chapter{Calculus and trigonometry}

\prerequisites{basic trigonometry.}

\section{Invitation: oscillations, measurement and trigonometry}

Trigonometry is all about measuring triangles and circles that contains them. Trigonometric functions can be used to describe oscillation, vibrations, and anything that's going in circles; therefore, they show up as a very important tool in physics. We'll also see that trigonometric functions can be used as a tool for integrating very complex integrals. So, we shall study them in this chapter.

As an overview, our trigonometric journey starts from analyzing the behavior of sine and cosine: two of the most important trigonometric functions. Then, we'll derive a polynomial expression for these functions. After which we'll see that these expressions come up when we're dealing with oscillating system; the simplest one being the simple harmonic oscillator, aka the spring-mass system. We'll also talk about how trigonometry can be used to describe things that go around in circles; thus, the Newton's law in radial coordinate systems. From there, we'd be able to derive some pretty well-known equations in circular motion. Then, we'll finish off with the wave equation and complex numbers.

\section{The behavior of sine and cosine}

\begin{figure}[ht]
    \centering
    \includegraphics{basiccalculusandtrigonometry/sineandcosinerelation.pdf}
    \caption{The relation between sine (in black) and cosine (in dashed)}
    \label{fig:sine_cosine_relation}
\end{figure}

Two of the most important trigonometric functions are sine and cosine. In calculus, we study the rate of change of these functions. Let's start with the sine function. When the sine function oscillates, its derivative also oscillate: $x = 0$, $\sin(x) = 0$, but the slope is rising up. When $x = \flatfrac{\cpi}{2}$, $\sin(x)$ reaches its peak at $1$, but it flattens down and the slope is zero. Later, it dips down at $x = \cpi$, $\sin(x) = 0$, but now the slope is going down. If we plot the slope of sine at each point (shown as dots in \cref{fig:sine_cosine_relation}), you'll see that it resembles a cosine curve. Maybe it as well be, but we'd have to mathematically prove that.

Let's start with the definition of the derivative given in \cref{eq:naive_definition_of_derivative},
\begin{equation}
    \odv*{\sin(x)}{x} = \lim_{h \appr 0}\frac{\sin(x + h) - \sin(x)}{h}.
\end{equation}
The sum of sines reads $\sin(x + h) = \sin(x)\cos(h) + \cos(x)\sin(h)$; therefore,
\begin{align}
    \odv*{\sin(x)}{x} &= \lim_{h \appr 0}\frac{\sin(x)\cos(h) + \cos(x)\sin(h) - \sin(x)}{h}
    \intertext{When $h \appr 0$, $\cos(h) \appr 1$}
    &= \lim_{h \appr 0}\frac{\sin(x) + \cos(x)\sin(h) - \sin(x)}{h} \\
    &= \cos(x)\lim_{h \appr 0}\frac{\sin(h)}{h}.
    \intertext{For small angles, $\sin(h) \approx h$, therefore}
    &= \cos(x)\lim_{h \appr 0}\frac{h}{h} = \cos(x).
\end{align}
In conclusion, the derivative of sine is just the cosine just as we predicted with the graph.

Similarly, the derivative of cosine can also be found using the method of increments and using the sum of cosines formula $\cos(x + h) = \cos(x)\cos(h) - \sin(x)\sin(h)$
\begin{align}
	\odv*{\ab(\cos(x))}{x} &= \lim_{h \appr 0}\frac{\cos(x + h) - \cos(x)}{h} \\
						   &= \lim_{h \appr 0}\frac{\cos(x)\cos(h) - \sin(x)\sin(h) - \cos(x)}{h} \\
						   \intertext{When $h \appr 0$, $\cos(h) = 1$}
						   &= \lim_{h \appr 0}\frac{\cos(x) - \cos(x) - \sin(x)\sin(h)}{h} \\
						   &= \lim_{h \appr 0}\frac{-\sin(x)\sin(h)}{h}
						   \intertext{For small angles of $h$, $\sin(h) \approx h$}
						   &= \lim_{h \appr 0}\ab(-\sin(x)) = -\sin(x) 
\end{align}
I.e., the derivative of cosine is the negative of sine. From those two alone, we can form a set of formulas for differentiating sines and cosines:
\begin{equation}	
	\begin{aligned}
		\odv*{(\sin(x))}{x} &= \cos(x) \\
		\odv*[order = 2]{\ab(\sin(x))}{x} &= \odv*{(\cos(x))}{x} = -\sin(x) \\
		\odv*[order = 3]{\ab(\sin(x))}{x} &= \odv*{(-\sin(x))}{x} = -\cos(x) \\
		\odv*[order = 4]{\ab(\sin(x))}{x} &= \odv*{(-\cos(x))}{x} = \sin(x)
	\end{aligned}
\end{equation}
We can see that the derivative of both sine and cosine cycles in pattern of four: $\sin(x), \cos(x), -\sin(x), -\cos(x)$. From these, we can use the rules developed in \cref{sec:basic_derivatives_and_integrals}, e.g., the chain rule, power rule, etc., in order to find the derivative of other trigonometric functions.

\section{Derivative of other trigonometric functions}

Note that I wouldn't be referring back to this section often. You can continue reading the next section straightaway. But for those of you who are still interested, I'd be glad to have you here. Because this is one of the places that we need the chain rule, product rule, and power rule to help us find the derivative. So it's a good exercise. Let's start with the reciprocals of trigonometric functions first: the cosecant ($\csc(x) = \frac{1}{\sin(x)}$),
\begin{align}
	\odv*{\csc(x)}{x} &= \odv*{\frac{1}{\sin(x)}}{x} \\
						&= \odv*{\frac{1}{u}}{x} \cdot \odv{u}{u} && \textrm{Chain rule: $\sin(x) = u$} \\
						&= \odv*{u^{-1}}{u} \cdot \odv{u}{x} \\
						&= -1u^{-2}\cdot\odv*{\sin(x)}{x} && \textrm{Power rule, $u = \sin(x)$} \\
						&= -\frac{1}{\sin^2(x)}\cos(x) && \textrm{$\odv*{\sin(x)}{x} = \cos(x)$} \\
						&= -\csc(x)\cot(x) && \textrm{$\cot(x) = \frac{\cos(x)}{\sin(x)}$}.
\end{align}
And the secant function ($\sec(x) = \frac{1}{\cos(x)}$),
\begin{align}
	\odv*{\sec(x)}{x} &= \odv*{\frac{1}{\cos(x)}}{x} \\
					  &= \odv*{\frac{1}{u}}{x} \cdot \odv{u}{u} && \textrm{Chain rule: $\cos(x) = u$} \\
					  &= \odv*{u^{-1}}{u} \cdot \odv{u}{x} \\
					  &= -1u^{-2}\cdot\odv*{\cos(x)}{x} && \textrm{Power rule, $u = \cos(x)$}\\
					  &= -\frac{1}{\cos^{2}(x)}\ab(-\sin(x)) && \textrm{$\odv*{\cos(x)}{x} = -\sin(x)$} \\
					  &= \sec(x)\tan(x).
\end{align}

For the tangent function ($\tan(x) = \frac{\sin(x)}{\cos(x)}$), we use the product rule and the results from earlier
\begin{align}
	\odv*{\tan(x)}{x} &= \odv*{\frac{\sin(x)}{\cos(x)}}{x} \\
					  &= \odv*{\sin(x)\sec(x)}{x} \\
					  &= \sin(x)\odv*{\sec(x)}{x} + \sec(x)\odv*{\sin(x)}{x} \\
					  &= \sin(x)\sec(x)\tan(x) + \sec(x)\cos(x) \\
					  &= \sin(x)\frac{1}{\cos(x)}\frac{\sin(x)}{\cos(x)} + \frac{1}{\cos(x)}\cos(x) \\
					  &= \ab(\frac{\sin(x)}{\cos(x)})^2 + 1 = \tan^2(x) + 1 = \sec^2(x)
\end{align}
The Pythagorean identity is used in the last line to convert $1 + \tan^2(x)$ to $\sec^2(x)$. Of course, we cannot miss its reciprocal function: $\cot(x) = \frac{1}{\tan(x)}$.
\begin{align}
	\odv*{\cot(x)}{x} &= \odv*{\frac{1}{\tan(x)}}{x} \\
					  &= \odv*{\frac{\cos(x)}{\sin(x)}}{x} \\
					  &= \odv*{\cos(x)\csc(x)}{x} \\
					  &= \cos(x)\odv*{\csc(x)}{x} + \csc(x)\odv*{\cos(x)}{x} \\
					  &= \cos(x)\ab(-\csc(x)\cot(x)) + \csc(x)\ab(-\sin(x)) \\
					  &= -\cos(x)\frac{1}{\sin(x)}\frac{\cos(x)}{\sin(x)} - \frac{1}{\sin(x)}\sin(x) \\
					  &= -\ab(\cot^2(x) + 1) = -\csc^2(x),
\end{align}
where we've again used the Pythagorean identity $\cot^2(x) + 1 = -\csc^2(x)$.

Altogether, I've summarized all the derivatives of trigonometric functions into \cref{tab:derivative_trigonometric_functions}.
\begin{table}[ht]
	\begin{center}
		\begin{tabular}{C | L}
			f(x) & \odv*{f(x)}{x} \\
			\hline
			\sin(x) & \cos(x) \\
			\cos(x) & -\sin(x) \\
			\tan(x) & \sec^2(x) \\
			\hline
			\csc(x) & -\csc(x)\cot(x) \\
			\sec(x) & \sec(x)\tan(x) \\
			\cot(x) & -\csc^2(x)
		\end{tabular}
	\end{center}
	\caption{The derivatives of trigonometric functions}
	\label{tab:derivative_trigonometric_functions}
\end{table}


\section{Power series of various trigonometric functions}

The approximation method that I'd be discussing are commonly taught to high schoolers under the name of \textbf{small angle approximation}, i.e., for $\theta \appr 0$, \index{small angle approximation}
\begin{gather}
    \sin(\theta) \approx \tan(\theta) \approx \theta \mathand \cos(\theta) \approx 1 - \frac{\theta^2}{2}
\end{gather}
These approximations are very useful for calculating limits, e.g.,\footnote{Normally these limits are evaluated using the squeeze theorem. However, I don't want the mathematical intricacies to disrupt the flow of our journey right now. The full theorem will be discussed later at the end of the book.}
\begin{equation}
    \lim_{\theta \appr 0}\frac{\sin(n\theta)}{\theta} = \lim_{\theta \appr 0}\frac{n\theta}{\theta} = n.
\end{equation}
Here, I want to focus on where these approximations really comes from: the power series development of trigonometric functions.

\subsection{Power series development for sine and cosines}
\label{sec:sine_cosine_power_series}

As we've seen in \cref{sec:function_equals_own_derivative}, every function has a power series expansion
\begin{equation}
    a_0 + a_1x^1 + a_2x^2 + a_3x^3 + \dots.
\end{equation}
In this section, we'll develop the power series expansion for sine and cosine. Let's start with the sine function,
\begin{align}
	\sin(x) = s_0 + s_1x^1 + s_2x^2 + s_3x^3 + \dots
\end{align}
Because $\sin(0) = 0$,
\begin{align}
	\sin(0) &= s_0 + s_1(0)^1 + s_2(0)^2 + s_3(0)^3 + \dots \\
	0 &= s_0;
\end{align}
thus,
\begin{equation}
	\sin(x) = s_1x^1 + s_2x^2 + s_3x^3.
\end{equation}
To proceed further, we must infuse more information about the behavior of sines into the power series. Namely, that sine is an odd function: $\sin(-x) = -\sin(x)$.
\begin{align}
	\sin(-x) &= -\sin(x) \\
	\begin{multlined}[b]
		s_1(-x)^1 + s_2(-x)^2 \\
		+ s_3(-x)^3 + s_4(-x)^4 + \dots
	\end{multlined} &=
	\begin{multlined}[t]
		-s_1x^1 - s_2x^2 \\
		- s_3x^3 - s_4x^4 + \dots 
	\end{multlined} \\
	\begin{multlined}[b]
		-s_1x^1 + s_2x^2 \\
		- s_3x^3 + s_4x^4 - \dots
	\end{multlined} &=
	\begin{multlined}[t]
		-s_1x^1 - s_2x^2 - \\
		s_3x^3 - s_4x^4 - \dots
	\end{multlined}
\end{align}
In the L.H.S., the sign oscillates between negative and positive. In order for the L.H.S. to match the R.H.S., the positive terms must vanish; therefore,
\begin{equation}
	\sin(x) = s_1x^1 + s_3x^3 + s_5x^5 + \dots.
\end{equation}
This is what we mean by \enquote{sine is an odd function}: there are only odd terms in its polynomial expansion.

To find $s_1$, we can get rid of the multiplier $x$ by taking the derivative of sine, which is cosine.
\begin{equation}
	\odv*{\sin(x)}{x} = \cos(x) = s_1 + 3s_3x^2 + 5s_5x^4 + \dots.
\end{equation}
Substitute $x = 0$,
\begin{align}
	\cos(0) = 1 &= s_1 + 3s_3(0)^2 + 5s_5(0)^4 + \dots \\
	1 &= s_1
\end{align}
Now that we know $s_1$, we can take the derivative of sine again to get a recurrence relation.
\begin{align}
	\odv*[order = 2]{\sin(x)}{x} &= 3\cdot 2s_3x + 5\cdot 4s_5x^3 + 7\cdot 6s_7x^5 + \dots \\
	-\sin(x) &= \frac{3!}{1!}s_3x + \frac{5!}{3!}s_5x^3 + \frac{7!}{5!}s_7x^5 + \dots \\
	-s_1x - s_3x^3 - s_5x_5 - \dots &= \frac{3!}{1!}s_3x + \frac{5!}{3!}s_5x^3 + \frac{7!}{5!}s_7x^5 + \dots
\end{align}
By matching terms with the same order,
\begin{equation}	
	\begin{aligned}
		-s_1 &= \frac{3!}{1!}s_3x, \\
		-s_3 &= \frac{5!}{3!}s_5x, \\
		-s_5 &= \frac{7!}{5!}s_7x. \\
			 &\vdots
	\end{aligned}
\end{equation}
From $s_1 = 1$, we get that $s_3 = -\frac{1}{3!}$, $s_5 = -\frac{1}{5!}$, $s_7 = -\frac{1}{7!}$ etc. The negative sign alternates every term. We can then summarize the whole thing as
\begin{align}
	\sin(x) &= \sum_{n = 0}^{\infty}\frac{(-1)^n}{n!}x^n \\
			&= x - \frac{1}{3!}x^3 + \frac{1}{5!}x^5 - \frac{1}{7!}x^7 + \dots,
\end{align}
which is the polynomial expansion of the sine function.

To develop the power series expansion for cosine, we can use the same method. But we can just take the derivative of the sine power series once to get cosine:
\begin{align}
	\odv*{\sin(x)}{x} &= \odv*{\ab(x - \frac{1}{3!}x^3 + \frac{1}{5!}x^5 - \frac{1}{7!}x^7)}{x} \\
	\cos(x) &= 1 - \frac{1}{2!}x^2 + \frac{1}{4!}x^4 - \frac{1}{6!}x^6 + \dots 
\end{align}

\subsection{Approximation of trigonometric functions}

One way a function can be approximated is by truncating its power series. To see why this is viable, consider the limit
\begin{align}
	\lim_{x \appr 0}\sin(x) = \lim_{x \appr 0}\ab(x - \frac{1}{3!}x^3 + \frac{1}{5!}x^5 - \frac{1}{7!}x^7 + \dots)
\end{align}
When $x \appr 0$, all the higher order degrees term vanishes; thus, 
\begin{gather}
	\lim_{x \appr 0}\sin(x) = x.
\end{gather}
It tells us that when $x \approx 0$, the sine function behaves like a linear function. Illustrated in \cref{fig:sine_small_angle},

\begin{figure}[ht]
	\centering
	\includegraphics{basiccalculusandtrigonometry/sinesmallangle.pdf}
	\caption{Comparison of $y = \sin(x)$ and $y = x$ near $x = 0$}
	\label{fig:sine_small_angle}
\end{figure}
When $x$ gets larger, the higher order terms in the power series contributes more and more, so we need to include them. An excellent approximation for the sine function when $x \in [-\cpi, \cpi]$ is a truncation at the fifth term:
\begin{equation}
	\sin(x) \approx x - \frac{1}{3!}x^3 + \frac{1}{5!}x^5 - \frac{1}{7!}x^7 + \frac{1}{9!}x^9,
\end{equation}
plotted in \cref{fig:sine_approximation}. It is extremely accurate in that range.
\begin{figure}
	\centering
	\includegraphics{basiccalculusandtrigonometry/sineapprox.pdf}
	\caption{An approximation of sine by truncating its power series at the fifth term.}
	\label{fig:sine_approximation}
\end{figure}

For cosine,
\begin{align}
	\lim_{x \appr 0}\cos(x) = \lim_{x \appr 0}\ab(1 - \frac{x^2}{2!} + \frac{x^4}{4!} - \dots) \approx 1 - \frac{x^2}{2!} \approx 1.
\end{align}
An acceptable approximation around zero is $1$ or $1 - \frac{x^2}{2}$. Similarly, an excellent approximation when $x \in [-\cpi, \cpi]$ is also a truncation at the fourth term:
\begin{equation}
	\cos(x) \approx 1 - \frac{1}{2!}x^2 + \frac{1}{4!}x^4 - \frac{1}{6!}x^6 
\end{equation}
plotted in \cref{fig:cosine_approximation}
\begin{figure}
	\centering
	\includegraphics{basiccalculusandtrigonometry/cosineapprox.pdf}
	\caption{An approximation of cosine by truncating its power series at the fourth term}
	\label{fig:cosine_approximation}
\end{figure}

\section{The harmonic oscillator}

Now that we have the mathematical foundations built, let's start tackling our first oscillatory system: the harmonic oscillator. A physical harmonic oscillator can be constructed using a spring attached to a mass, illustrated in \cref{fig:simple_harmonic_oscillator_setup}. The stiffness of the spring can be described by a stiffness constant $k$. The higher the $k$, the stiffer the spring. The point $x = 0$ is set at the spring's equilibrium point, i.e., it's neither stretched or compressed. The further it is away from this equilibrium point, the more force the spring acts on the object to restore itself into the equilibrium position. By experiment, the force acted on the object for small $x$ is a $kx$; the negative sign suggests that the force acts on the opposite direction from the direction of the object. Now that we know the force, we can solve for the position function. The Newton's equation, $F = m\odv[ord = 2]{x}{t}$ then becomes
\begin{align}
	-kx &= m\odv[ord = 2]{x}{t} \\
	\odv[ord = 2]{x}{t} &= -\frac{k}{m}x \label{eq:simple_harmonic_oscillator_newton}
\end{align}
For simplicity, let's set $c^2 = \frac{k}{m}$ where $c \in \mathbb{R}$.
\begin{figure}[ht]
	\centering
	\includegraphics{basiccalculusandtrigonometry/harmonicoscillatorsetup.pdf}
	\caption{Setup of a simple harmonic oscillator}
	\label{fig:simple_harmonic_oscillator_setup}
\end{figure}

I'd like to focus on the qualitative part of this system before we start to solve it analytically. The spring system is an oscillatory system. At $t = 0$, the mass is at $x = x_0$. For simplicity's sake, let $v(t = 0) = 0$. The spring's restoration force pulls on the object, accelerating it. As the object gets closer to the equilibrium, the restoration force gets smaller. The continuously accelerates until it overshoots the equilibrium position; after which, the spring is compressed and tries to push back on the object. This continues until the object finally stops at the other end. Then, the spring pulls back, the object overshoots the equilibrium position again and stops at the original position, completing one oscillation cycle.

From that description alone, we can already extract some key features to the spring system.
\begin{enumerate}
	\item The object reaches zero velocity when the spring is fully stretched or compressed.
	\item The acceleration is highest when the spring is fully stretched or compressed.
	\item The velocity is highest when the object passes through the equilibrium position. 
\end{enumerate}
As we'll see, these features will all be reflected in the result.

There are actually two ways of solving \cref{eq:simple_harmonic_oscillator_newton}. The first one is to just guess the solution straightaway! It's called the \textbf{ansatz method}, which is commonly taught in Massachusets Institute of Technology (MIT) \index{ansatz method}

\section{Polar coordinate systems}

% \chapter{Significance of calculus}
\label{sec:significanceofcalculus}

\begin{abstract}
    I've massively restructured the contents of this chapter from the normal calculus textbook. Earlier in \cref{sec:basicdifferentialequations}, we see that calculus have much significance in kinematics. We'll be discussing about that later in \cref{sec:ode1}; however, I put a few of the worked out kinematics problems in the beginning of this chapter. After that, we'll be applying calculus in various other problems. Including
    \begin{enumerate}[noitemsep]
        \item Higher dimensional quantities: area and volumes
        \item Optimization
        \item Root finding algorithm
    \end{enumerate}

    Also, I will mention various techniques of integration needed along the way. Mainly, substitution of variables.

    \prerequisites{Basic derivatives and integrals (\cref{sec:basicderivativesandintegrals}), free body diagram writing}
\end{abstract}

\section{Newton's fluxion notation}
\label{sec:newtons_fluxion_notation}

Before moving to further examples in kinematics, I'd like to introduce another notation called the \textbf{Newton's fluxion notation}\index{notation for derivatives!Newton's fluxion notation} or, the \textbf{dot notation}\index{dot notation|see {notation for derivatives}}. \emph{This notation is used only when the derivative is took w.r.t. time}. It places a dot over the variables, e.g., the first derivative of position $r$ w.r.t. time is $\dot{x}$.

Higher derivatives notation is written with more dots, e.g., the second derivative of position $r$ w.r.t. is $\ddot{r}$. The third derivative is $\dddot{r}$, forth derivative, $\ddddot{r}$ and so on.

\section{Further examples of calculus in physics and kinematics}

\subsection{Block sliding down a ramp with friction}

\begin{figure}[ht]
    \centering
    \includegraphics{geometricalsignificance/blockslidingdownaramp}
    \caption{A block sliding down a rough ramp (i.e. with friction) angled $\theta$ relative to the ground}
    \label{fig:blockslidingdownaramp}
\end{figure}
Illustrated in \cref{fig:blockslidingdownaramp}, there are two axis of motion: perpendicular and parallel to the block's expected motion. From what we know, there are the gravity $mg$ that pulls the block straight down and the friction force $f_r$ acting against the block's expected motion in the parallel axis. The problem constraints use along the perpendicular axis: the block can't move along that perpendicular axis \emph{because there's a ramp in the way}. Therefore, the ramp must act a force $N$ on the block.

We want to find the equation of motion for this block that's sliding down a ramp with friction. As far as the perpendicular axis goes, we don't have to worry about that one because nothing is moving there anyways. On the parallel axis, there's the $f_r$ and $mg\sin(\theta)$.\footnote{Just decompose $mg$ into its parallel and perpendicular axis. Using basic trigonometry is enough.} If we set the direction of the parallel axis to pointing downwards along the block's movement, we get that the total force is
\begin{equation*}
    F_{n} = mg\sin(\theta) - f_r\footnote{$F_n$ for $F_{\textrm{net}}$ or, total force.}.
\end{equation*}
By the Newton's second law,
\begin{align*}
    m\dv{t}(\dv{x}{t}) = mg\sin(\theta) - f_r \\
    \dv{t}(\dv{x}{t}) = g\sin(\theta) - \frac{f_r}{m}.
\end{align*}
$g\sin(\theta) - \flatfrac{f_r}{m}$ doesn't change with time, let's name it $\kappa$. The equation then becomes
\begin{align}
    \dv{t}(\dv{x}{t}) &= \kappa \label{eq:blockslidingdownarampaux}\\
    \int\dd(\dv{x}{t}) &= \int\kappa\dd{t} \nonumber\\
    \dv{x}{t} &= \kappa t \nonumber \\
    \int\dd{x} &= \int\kappa t\dd{t} \nonumber\\
    x(t) &= \left(g\sin(\theta) - \frac{f_r}{m}\right)\frac{t^2}{2}.
\end{align}
which \emph{is} our equation of motion. Notice, \cref{eq:blockslidingdownarampaux} literally has the same form as \cref{eq:accelerationfromgravity} that we derived from ``ball dropped from a building''. And indeed, it should be the same because it's just a thing that's under a constant acceleration. 

\subsection{One-dimensional movement with drag forces}

The free body diagram is illustrated in \cref{fig:projectileswithdragforce}. There's the gravity $m\vv{g}$ pulling the ball down, and drag force $kf_r(\vv{v})$. However, drag is a complex thing. There is no such thing as an ``exact drag function'' because drag depends on so many variables, e.g., air viscosity, air compressibility, object's shape, surface's friction, just to name a few. Therefore, the drag function $f_r(\vv{v})$ is a \emph{simplified model}, not the real thing.\index{drag function (simplified model)}

We shall model the drag based on two assumptions. \begin{enumerate*}[label = \roman*.)]
    \item the drag force should depend on the velocity : the faster, the more drag. And,
    \item any function can be approximated using the power series expansion (also discussed in \cref{sec:afunctionthatisitsownderivative}):
\end{enumerate*}
\begin{equation*}
    f_r(\vv{v}) = a_0 + a_1\vv{v} + a_2\vv{v}^2 + a_3\vv{v}^3 + \dots.
\end{equation*}
Considering only the first three terms should be enough. We know that $a_0$ must be $0$, because otherwise our object would just accelerate all the time, which is no good. Therefore, there can only be $a_1\vv{v} + a_2\vv{v}^2$. Newton's second law reads
\begin{equation*}
    m\dv{t}(\dv{x}{t}) = a_1\dv{x}{t} + a_2\left(\dv{x}{t}\right)^2.
\end{equation*}
Using fluxion notation,
\begin{align*}
    m\dv{\dot{x}}{t} &= a_1\dot{x} + a_2\dot{x}^2 \\
    \dv{\dot{x}}{t} &= \frac{a_1}{m}\dot{x} + \frac{a_2}{m}\dot{x}^2
\end{align*}
For convenience, let $\flatfrac{a_1}{m} = p$ and $\flatfrac{a_2}{m} = q$. The equation reads
\begin{equation}
    \dv{\dot{x}}{t} = p\dot{x} + q\dot{x}^2. \label{eq:quadraticdrageq}
\end{equation}

It is obvious that both $p$ and $q$ must be negative, otherwise the object would accelerate forward with the velocity. Frankly, \cref{eq:quadraticdrageq} is not possible to solve using the techniques that we have now. I'll revisit this exact differential equation later in \cref{eq:advtechniquesrationalfunctions}. For now, we shall deal with a simpler equation by considering two cases: only linear drag and only quadratic drag.

\subsubsection{Motion with just linear drag}

You don't really see linear drag in real life. It's mostly drag in moving liquid, e.g., a fish swimming in the water. \Cref{eq:quadraticdrageq} simplifies to
\begin{equation*}
    \dv{\dot{x}}{t} = p\dot{x}.
\end{equation*}
I shall set $v_0$ as the initial velocity and $x_0$ as the initial position. There are two methods of solving this. First, by separating variables using analytical methods.
\begin{align}
    \int\frac{1}{\dot{x}}\dd{\dot{x}} &= p\int\dd{t} \nonumber\\
    \ln(\dot{x}) + C &= pt. \label{eq:motionlineardrag1}
\end{align}
To find out what this integration constant should be, we have to use the initial condition $t = 0 \implies \dot{x} = v_0$.
\begin{align*}
    \ln(v_0) + C &= 0 \\
    C &= -\ln(v_0).
\end{align*}
\Cref{eq:motionlineardrag1} then becomes
\begin{align}
    \ln(\dot{x}) - \ln(v_0) &= pt \nonumber\\
    \dot{x} &= v_0\e^{pt}. \label{eq:motionlineardrag2}
\end{align}

\begin{figure}
    \centering
    \includegraphics{geometricalsignificance/motionlineardragv}
    \caption{Plot of \cref{eq:motionlineardrag2} with various $p$.}
    \label{fig:motionlineardragv}
\end{figure}
\Cref{fig:motionlineardragv} plots \cref{eq:motionlineardrag2} with various $p$. Notice that when $p = 0$, \cref{eq:motionlineardrag2} is just a straight line $\dot{x} = v_0$. The more negative $p$ is, the faster it slows down, as illustrated. That is why sometimes, we call $p$ the \textbf{dampening factor}\index{dampening factor}. Also, $v_0$ here just scales the graph in the $\dot{x}$ direction.

To find $x(t)$, we rewrite \cref{eq:motionlineardrag2} as
\begin{equation*}
    \dv{x}{t} = v_0\e^{pt}.
\end{equation*}
Then,
\begin{align}
    \dd{x} &= v_0\e^{pt}\dd{t} \\
    x + C_1 &= v_0\int \e^{pt}\dd{t}. \label{eq:motionlineardrag3}
\end{align}

Here, I shall introduce an integration technique called \textbf{change of variables}\index{change of variables!naive}, commonly known as $u$-substitution. We'll formally come back to this topic later in \cref{sec:changeofvariables}. Basically, it's a way to convert integrals that we don't recognize into an easier integral. It's better if I just show the examples. We don't know the antiderivative of $\e^{pt}$ in \cref{eq:motionlineardrag3}, however we know the antiderivative of $\e^{u}$. So let's convert $\e^{pt}$ into that form. By letting a dummy variable $u = pt - v_0$, we have to convert $\dd{u}$ into $\dd{t}$ aswell.
\begin{align*}
    u &= pt - v_0 \\
    \dv{u}{t} &= \dv{t}(pt - v_0) \\
    \frac{1}{p}\dd{u} &= \dd{t}.
\end{align*}
Then, substituting $\dd{t} = \frac{1}{p}\dd{u}$ and $u = pt$ into \cref{eq:motionlineardrag3}, it reads
\begin{align}
    x + C_1 &= v_0\int\e^{u}\left(\frac{1}{p}\dd{u}\right) \nonumber\\
    x + C_1 &= \frac{v_0}{p}\e^{pt}. \label{eq:motionlineardrag4}
\end{align}

We also have to take care of the $C_1$. Using $t = 0 \implies x = x_0$,
\begin{align*}
    x_0 + C_1 &= \frac{v_0}{p}\e^{p\cdot(0)} \\
    C_1 &= \frac{v_0}{p} - x_0.
\end{align*}
Then, plug it into \cref{eq:motionlineardrag4}. The equation becomes
\begin{align}
    x + \frac{v_0}{p} - x_0 &= \frac{v_0}{p}(\e^{pt}) \nonumber\\
    x &= x_0 + \frac{v_0}{p}(\e^{pt} - 1). \label{eq:motionlineardrag5}
\end{align}

I've plotted \cref{eq:motionlineardrag5} with varying $v$ in \cref{fig:motionlineardragx1} and varying $p$ in \cref{fig:motionlineardragx2}. The graph should match with what you expect intuitively: higher $v$ will get you further, and the less the drag, the further you'll get.

\begin{figure}[ht]
    \centering
    \begin{subfigure}[l]{0.45\textwidth}
        \centering
        \includegraphics{geometricalsignificance/motionlineardragx1}
        \caption{with varying $v$, setting $p = -1$}
        \label{fig:motionlineardragx1}
    \end{subfigure}
    \begin{subfigure}[r]{0.45\textwidth}
        \centering
        \includegraphics{geometricalsignificance/motionlineardragx2}
        \caption{with varying $p$, setting $v = 2$}
        \label{fig:motionlineardragx2}
    \end{subfigure}
    \caption{Plot of \cref{eq:motionlineardrag5}, setting $x_0 = 0.5$.}
\end{figure}

Notice, this motion has a clear upper limit. \emph{If $p \in \mathbb{R}^-$}, $\lim_{t \appr \infty}e^{pt} = 0$. Taking the limit as $t \appr 0$ on both sides of \cref{eq:motionlineardrag5}, we get
\begin{align*}
    \lim_{t \appr \infty} x(t) &= \lim_{t \appr \infty}\left(x_0 + \frac{v_0}{p}(e^{pt} - 1)\right). \\
    &= x_0 + \frac{v_0}{p}\cancelto{0}{\lim_{t \appr 0}\left(e^{pt}\right)} - \frac{v_0}{p} = x_0 - \frac{v_0}{p},
\end{align*}
which when $p \in \mathbb{R}^-$ and $v_0\in\mathbb{R}^+$, the motion procedes forward, then gradually slows down and stops at $x_0 - \frac{v_0}{p}$ which is more than $x_0$.

\subsubsection{Motion with just quadratic drag}

\begin{multicols}{2}
This equation is even easier than linear drag, so I'd leave out some steps. \Cref{eq:quadraticdrageq} simplifies to
\begin{align*}
    \dv{\dot{x}}{t} &= q\dot{x}^2 \\
    \int \frac{1}{\dot{x}^2}\dd{\dot{x}} &= q\int\dd{t} \footnote{Hint: $\flatfrac{1}{\dot{x}^2} = \dot{x}^{-2}$}\\
    - \frac{1}{\dot{x}} + C = qt.
\end{align*}
Taking care of $C$: use $t = 0 \implies \dot{x} = v_0$.
\begin{align*}
    - \frac{1}{v_0} + C &= q\cdot 0 \\
    C &= \frac{1}{v_0}.
\end{align*}
Therefore,
\begin{align}
    -\frac{1}{\dot{x}} + \frac{1}{v_0} &= qt \nonumber\\
    \dv{t}{x} &= \frac{1 - qv_0t}{v_0} \nonumber\\
    x &= v_0\int\frac{1}{1 - qv_0t} \dd{t}. \label{eq:motionquadraticdrag1}
\end{align}
The structure of the integral is similar to $\flatfrac{1}{t}$. Therefore, let $u = 1 - qv_0t$. Then,
\begin{align*}
    \dv{u}{u} &= \dv{u}(1 - qv_0t) \\
    1 &= -qv_0\dv{t}{u} \\
    \dd{t} &= -\frac{1}{qv_0}\dd{u}.
\end{align*}
Substitute $u = 1 - qv_0t$ and $\dd{t} = -\frac{1}{qv_0}\dd{u}$ into \cref{eq:motionquadraticdrag1}:
\begin{align}
    x &= v_0 \int \frac{1}{u}\left(-\frac{1}{qv_0}\dd{u}\right) \nonumber\\
    &= -\frac{1}{q}\int \frac{1}{u}\dd{u} \nonumber\\
    x(t) &= -\frac{1}{q}\ln(1 - qv_0t) + C_1. \label{eq:motionquadraticdrag2}
\end{align}
Taking care of $C_1$: use $t \appr 0 \implies x(t) = x_0$.
\begin{align*}
    x_0 &= -\frac{1}{q}\ln(1 - qv_0\cdot(0)) + C_1 \\
    x_0 &= C_1
\end{align*}
Plug this into \cref{eq:motionquadraticdrag2} to get the final answer:
\begin{equation*}
    x(t) = -\frac{1}{q}\ln(1 - qv_0t) + x_0.
\end{equation*}
\end{multicols}

Surprisingly, quadratic drag does not have upper position bounds. A bit more thought would reveal that when $v < 0$, the quadratic drag $f_r(v) = a_2v^2$ is smaller thasecn $f_r(v) = a_1v$. Thus, it should makes sense that quadratic doesn't have bounds, but linear has an upper bound.

\subsection{Time of meteor collision from great height}

Illustrated in \cref{fig:meteorfallingfromgreatheight}\footnote{Not to scale.}, a meteor is falling from height $h$ above the Earth. Let's find the time that it'd take to hit the Earth.
\begin{figure}[ht]
    \centering
    \includegraphics{geometricalsignificance/meteorfallingfromgreatheight}
    \caption{An illustration of a meteor falling to the Earth from height $h$.}
    \label{fig:meteorfallingfromgreatheight}
\end{figure}
The meteor has the mass $m$, falling from height $h$ above the ground. The Earth has mass $M$, radius $R$. Let's denote the meteor's position relative to the Earth's center with $r$. The initial condition of the meteor is $r(0) = h + R$, and $\dot{r}(0) = v_0$ For simplicity sake, let the meteor be a mass point: its radius is zero, and the Earth is very massive compared to the meteor so it doesn't move with the meteor's gravitational attraction. Given by Newton's law of gravitational attraction, the Earth is pulling the meteor by
\begin{equation*}
    F = G\frac{Mm}{r^2}.
\end{equation*}
Therefore, Newton's second law on the meteor reads
\begin{equation*}
    G\frac{Mm}{r^2} = m\dv{t}(\dv{r}{t})\footnote{Notice that here, $r$ also changes with time.}.
\end{equation*}

To frame the problem mathematically, we want to find an equation of motion of the meteor. Then, find the time it takes for the meteor to travel from $h + R$ (initial position) to $R$ (the ground).

The $m$ on both sides cancel. For convenience, let $\kappa = GM$
\begin{equation}
    \frac{\kappa}{r^2} = \dv{\dot{r}}{t}. \label{eq:meteorfallingfromgreatheight1}
\end{equation}
Solving this equation is not at all trivial: there are \emph{three} variables, i.e., $r$. $\dot{r}$, and $t$; however, an equation only has two sides. We can't possibly separate these variables. Here, I shall introduce a technique for solving this kind of differential equation. With the chain rule,
\begin{equation*}
    \dv{\dot{r}}{t} = \dv{\dot{r}}{r} \times \dv{r}{t} = \dot{r}\dv{\dot{r}}{r}:
\end{equation*}
we converted an expression that's dependent on other variable $t$ to be dependent on a lower derivative $r$ instead! Then $t$ is removed, or rather, hidden. You can interpret $\dv*{\dot{r}}{r}$ as the velocity at any given distance away from the Earth. Plug this into \cref{eq:meteorfallingfromgreatheight1}, we get
\begin{align}
    \frac{\kappa}{r^2} &= \dot{r}\dv{\dot{r}}{r} \nonumber\\
    \kappa\int\frac{1}{r^2}\dd{r} &= \int\dot{r}\dd{\dot{r}} \nonumber\\
    \frac{\kappa}{r} + C &= \frac{\dot{r}^2}{2} \label{eq:meteorfallingfromgreatheight2}
\end{align}
The term $+C$ is going to be a different because now, we don't have a $t$ to fix our initial condition. However, we know that when $\dot{r} = v_0$, $r = h + R$. Therefore,
\begin{align*}
    -\frac{\kappa}{h + R} + C &= \frac{v_0^2}{2} \\
    C &= \frac{v_0^2}{2} - \frac{\kappa}{r}.
\end{align*}
However, the structure of $C$ is quite complicated so, I wouldn't substitute it in yet until we get our final answer. Continuing with \cref{eq:meteorfallingfromgreatheight2}, we turn the $\dot{r}$ into the Leibniz's notation:
\begin{align*}
    \frac{\kappa + Cr}{r} &= \frac{1}{2}\left(\dv{r}{t}\right)^2. \\
    \sqrt{2}\sqrt{\frac{\kappa + Cr}{r}} &= \dv{r}{t} \\
    \int\dd{t} &= \sqrt{2}\int\sqrt{\frac{r}{\kappa + Cr}}\dd{r}
\end{align*}

Unfortunately, this integral is very hard to solve. But it is possible, and the solution to this integral is
\begin{equation*}
    \frac{C}{\kappa^{\flatfrac{3}{2}}\sqrt{r}}\sqrt{\frac{r}{\kappa r + C}}\sqrt{\frac{\kappa r}{C} + 1} \left(\sqrt{\kappa r}\sqrt{\frac{\kappa r}{C} + 1} - \sqrt{C}\sinh^{-1}\left(\sqrt{\frac{\kappa x}{C}}\right)\right),
\end{equation*}
which is quite a nightmare, but we will get back to this in the far far future.

\subsection{One-dimensional simple harmonic motion}

A harmonic oscillator is described by an object that's under a force $F(x) = -kx$, which is a function of position. Here, we can use Newton's second law straightaway:
\begin{align}
    m\dv{t}(\dv{x}{t}) &= -kx \\
    \dv{t}(\dv{x}{t}) &= -\frac{k}{m}x. \label{eq:simpleharmonicmotion1}
\end{align}

There are actually three ways of approaching this, which I shall get you through all three.

\subsubsection{Ansatz method of solving differential equations}

The first one is called the \emph{ansatz}\index{ansatz} method, which is commonly taught in the MIT university. Ansatz means to make assumptions; this method assumes the solution of the differential equation then finalize it later. In this case, you have to ask yourself what function when differentiated twice w.r.t. time gives the function itself times a constant? Well, there are two functions which satisfies this: $\sin(t)$ and $\cos(t)$.

Notice that this differential equation is \emph{linear}. That means, if $f(t)$ is a solution, and $g(t)$ is a solution, then $f(x) + g(x)$ is also a solution. Therefore, our ansatz might look something like $\sin(t) + \cos(t)$. Our ansatz when differentiated twice must have a constant times itself. A pretty general ansatz that one might think of is $A\sin(\omega t) + B\cos(\omega t)$.

There's a reason why I used $\omega$ as a variable. $\omega$ suggests the angular speed of the $\sin(t)$ and $\cos(t)$. It will be evident later why it's the angular speed.

Now, with the ansatz in place, it's time to find $A$, $B$, and $\omega$. We can do that by just substituting the ansatz into \cref{eq:simpleharmonicmotion1}:
\begin{equation*}
    \dv{t}(\dv{t}(A\sin(\omega t) + B\cos(\omega t))) = -\frac{k}{m}\left(A\sin(\omega t) + B\cos(\omega t)\right).
\end{equation*}
We can then use the chain rule to the L.H.S. It's left as an exercise to the reader to verify that this is true.
\begin{align*}
    -A\omega^2\sin(\omega t) -B\omega^2\cos(\omega t) &= -\frac{k}{m}A\sin(\omega t) -\frac{k}{m}B\sin(\omega t) \\
    A\omega^2\sin(\omega t) + B\omega^2\cos(\omega t) &= \frac{k}{m}A\sin(\omega t) + \frac{k}{m}B\sin(\omega t).
\end{align*}
Here, we can match the coefficient and separate our equation into two parts:
\begin{gather*}
    A\omega^2\sin(\omega t) = \frac{k}{m}A\sin(\omega t), \\
    B\omega^2\cos(\omega t) = \frac{k}{m}B\cos(\omega t).
\end{gather*}
Both equations say that $(\flatfrac{k}{m})^{\flatfrac{1}{2}}$ must be equal to $\omega$. However, notice that both equations does not say anything about $A$, or $B$. That is simply because we can choose $A$ and $B$ ourselves; it's a free parameter. Our final solution is then
\begin{equation*}
    x(t) = A\sin(\sqrt{\frac{k}{m}}t) + B\cos(\sqrt{\frac{k}{m}}t).
\end{equation*}

\subsubsection{The series expansion method}

\subsection{Harmonic motion in two-dimensions: the pendulum}


\subsubsection{The repeated integration method}


\subsection{Damped harmonic motion}

\section{Conservation laws}

\section{Volumes of solids}

\section{The amount of real zeroes in a cubic equation}

\section{Optimization problems}

\section{Principle of least action}

\section{Tangent to a curve}

\section{Newton's root finding algorithm}
% \chapter{Series}

\section{Sequences}

\section{Geometric series: zeno's paradox}

\section{Convergence and divergence}

\subsection{Absolute convergence}

\section{Convergence test}

\section{Harmonic motion: the need of functions approximation}

\section{Taylor series}
% \chapter{Basic techniques of integration}
\label{sec:techniquesofintegration}

\begin{abstract}
    This chapter explores various techniques you can use to integrate stuffs. You still have to keep in mind that even though this chapter is very abstracted away, \emph{integrals still have its geometric meaning: area under the graph.} This geometric interpretation is going to somewhat appear quite often throughout the chapter, so be aware of it.

    Also, some integrals are definite: they have clear boundaries of integrations. You also have to keep track of what variable is bounded by that bound, otherwise you might end up evaluating incorrectly.
\end{abstract}

Note from the writer: As a calculus teacher myself, I usually don't teach this topic but leave it as an exercise. Techniques of integrations come naturally with much experiences. Therefore, I list a table of integrations at the end of the chapter for the physicists out there.

\section{Change of variables}
\label{sec:changeofvariables}

Change of variables, or integration by substitution is a method to convert an unknown integrals into a more familiar one. Consider $\int x\e^(x^2)\dd{x}$. This integral might seem daunting at first but, notice that the derivative of $x^2$ is exactly $x$. I shall introduce a new variable $u$ where $u = x^2$. If we take the derivative w.r.t. $x$ on both sides,
\begin{align*}
    \dv{u}{x} = 2x \\
    \dv{u} = 2x\dd{x}.
\end{align*}

\section{Integrals of inverse functions}

\section{Integrals of symmetric functions}

\subsection{Even and odd functions}

\subsection{Functions with axial and spherical symmetry}

\section{Integration by part}

\section{Recurrence relations and reduction formulas}

\section{Working with complex numbers and exponentials}

\section{Trigonometric substitution}

\section{Rational functions}

\subsection{Partial fraction decomposition}

\subsection{Quadratic under radicals}

\subsection{Ostrogradski method}

\section{Feynman's trick for integration}

\section{Polar integration}

\section{Gaussian integral}

\section{Cauchy's formula for repeated integrations}

\section{Numerical integration}
% \chapter{Advanced techniques of integration}
\label{sec:advancedtechniquesofintegration}

\begin{abstract}
    This is the chapter that explores more advanced techniques of integrations. You could go on with your life skipping this chapter and it'd still be fine. However, for the curious, there are some very interesting maths in here, so be sure to check this chapter out before going on to the next part of the book.
\end{abstract}

\section{Rational functions}
\label{eq:advtechniquesrationalfunctions}

\subsection{Quadratics denominator}

An integral in the form
\begin{equation}
    \int \frac{1}{ax^2 + bx + c}\dd{x}
\end{equation}
can be evaluated by completing the squares:
\begin{equation}
    ax^2 + bx + c = a\left(x + \frac{b}{2a}\right)^2 + \left(c - \frac{b^2}{4a}\right)
\end{equation}

\subsection{Denominator quadratics under radicals}

\section{Mathematical interlude: trigonometric substitution}

The trigonometric substitution technique earlier works for all the Pythagorean identity. The most used ones being
\begin{equation*}
	1 - \cos^2\theta = \sin^2\theta, \quad 1 + \tan^2\theta = \sec^2\theta \mathand 1 - \sec^2\theta = \tan^2\theta.
\end{equation*}
They're used to take terms outside the square root, whether it'd be $\sqrt{a^2 - x^2}$, $\sqrt{a^2 + x^2}$, or $\sqrt{x^2 - a^2}$. Now let's go through these one by one.

\paragraph{Integrals with $\sqrt{a^2 - x^2}$:} Let $x = a\sin\theta$; therefore, $\theta = \arcsin\ab(\frac{x}{a})$, and
\begin{equation}
	\sqrt{a^2 - x^2} = a\sqrt{1 - \sin^2\theta} = a\sqrt{\cos^2\theta} = a\cos\theta.
\end{equation}
The differential element $\odif{x}$ transforms into
\begin{align}
	\odv{x}{\theta} &= \odv*{\ab(a\sin\theta)}{\theta} \\
	\odif{x} &= a\cos\theta\odif{\theta}.
\end{align}
\begin{exmp}{$\int\frac{\sqrt{4 - x^2}}{x^2}\odif{x}$}{}
	This integral has a square root of the form $\sqrt{a^2 - x^2}$ where $a = 2$; thus, substitute in $x = 2\sin\theta$. Then,
	\begin{align}
		\odv{x}{\theta} &= \odv*{\ab(2\sin\theta)}{\theta} \\
		\odif{x} &= 2\cos\theta\odif{\theta}
	\end{align}
	The integral then becomes
	\begin{align}
	\int\frac{\sqrt{4 - 4\cos^2\theta}}{4\sin^2\theta}2\cos\theta\odif{\theta} &= \int \frac{2\sqrt{1 - \cos^2\theta}}{4\sin^2\theta}2\cos\theta\odif{\theta} \\
	&= \int\frac{2\sin^2\theta}{4\sin^2\theta}{2\cos\theta}\odif{\theta} \\
	&= \int\cos\theta\odif{\theta} = \sin\theta
	\end{align}
	Since $x = 2\sin\theta$, $\sin\theta = \frac{x}{2}$, our integral evaluates to $\frac{x}{2} + C$.
\end{exmp}

\paragraph{Integrals with $\sqrt{a^2 + x^2}$:} Let $x = a\tan\theta$; therefore, $\theta = \arctan\ab(\frac{x}{a})$, and
\begin{equation}
	\sqrt{a^2 + x^2} = a\sqrt{1 + \tan^2\theta} = a\sqrt{\sec^2\theta} = a\sec\theta.
\end{equation}
By \cref{tab:derivative_trigonometric_functions}, the differential element $\odif{x}$ transforms into
\begin{align}
	\odv{x}{\theta} &= \odv*{\ab(a\sec\theta)}{\theta} \\
	\odif{x} &= a\sec\theta\tan\theta.
\end{align}

\subsection{Feynman's trick for integration}

Formerly known as Leibniz's rule\index{Leibniz's integral rule} but popularized by Richard Feynman, is a very powerful integral technique that allows you to differentiate the function in the integral sign. Sadly, it only works for definite integrals. However, it's still a very nice trick to have just in case.

The idea is: integrals and derivatives can be swapped in 

\section{Integrals of other trigonometric functions}


% \part{The applications}

% \chapter{Ordinary differential equations I: basic concepts}
\label{sec:ode1}

\section{Basic example: projectile motion}
% \include{chapters/multipleintegrals}
% \include{chapters/numericalanalysis}
% \include{chapters/signalanalysis}

\part{The extensions}

\include{chapters/discretecalculus}
\include{chapters/fractionalcalculus}
\include{chapters/quantumcalculus}

\part{The fundamentals, reimagined: real analysis}

\chapter{Constructing the real numbers}
\index{real analysis}

\begin{abstract}
    \prerequisites{intuitions of set theory, basic discrete math}
\end{abstract}

As far as the ``calculus'' part of this book goes, it doesn't really delve deep into the proof: the backbones of how structures work together. For example, how do you know that in the definition of derivatives, if $h \appr 0$, the value actually converges to something. To be frank, real analysis is quite abstracted away from the physical reality therefore, it's a bit dry. This is quite the double edged sword of math. With real analysis, we have the power to tell clearly if something is true or not. However, most times it's mistakenly used to the bones: too abstract that the learner does not have any clear concepts leftover. All the rest is just some meaningless mathematical notation that's floating in the air. And I don't want that.

The goal for the real analysis part of this book is to provide an enjoyable experience delving in to the proofs behind the backbones of calculus. Therefore, I shall try to illustrate everything with diagrams so it's simple to visualize and not too abstracted away from reality. Now that you know my intentions, let's start.

\section{The mindset of real analysis}

Before we study the reals, we must know the mindset of real analysis first. Analysis is used to generalize and study the \emph{exact} behaviors of mathematical entities. In real analysis, we study the \emph{reals}. Most of the stuffs in mathematics were built way before real analysis. However, it's not rigorous and it's prone to error. Here, real analysis comes to play.

We \emph{abstract} properties of mathematical identities away from the numbers, and we generalize it. But we can't just choose everything, we must be very wise. The properties that we select to be true are called \textbf{axioms}\index{axioms} After all the decision has been done, we must find the most general mathematical entity that satisfies it. And thus, we shall begin with the most basics of analysis: set theories.

\section{The Zermelo–Fraenkel set theory}
\index{Zermelo-Fraenkel set theory}

In here, we shall explore what's the backbones of sets that will lead to the mechanics of numbers. And here arises the set theory. Firstly, a \index{set}\textbf{set} is a group of things, whether it be mathematical entities or real world objects. If two sets contains the same elements, then it's the same set. That means, set does not care about permutation. A wiser way to state this is
\begin{axiom}{Axiom of Extensionality}{}
    Two sets are the same if they have the same elements.
    \begin{equation}
        \forall X\forall Y[\forall z(z \in X \iff z \in X) \implies X = Y].
    \end{equation}
    Translation: Set $X$ and $Y$ will be equal iff for all elements $z$, $z$ is in both $X$ and $Y$.
\end{axiom}
which just means that ``\emph{A set is uniquely determined by its members}''.

Then, we also have to define that a set cannot have the same elements that is,
\begin{axiom}{Axiom of foundation}{}
    Every non-empty set $x$ contains a member $y$ such that $x$ and $y$ are \emph{disjoint}.
    \begin{equation}
        \forall x [x \neq \emptyset \implies \exists y((y \in x) \land (y \cup x) = \emptyset)]
    \end{equation}
    Translation: For all non-empty set $x$, there exists $y$ where both $y$
\end{axiom}

\part{Beyond imagination: complex analysis}

\include{chapters/treatmentofcomplexnumbers}

\appendix

\chapter{Fundamental of physics}
\label{appendix:fundamentalofphysics}
\chapter{The binomial theorem}
\label{appendix:binomialexpansion}

\backmatter

\printindex
\printbibliography

\end{document}
