\chapter{Basic techniques of integration}
\label{sec:techniquesofintegration}

\begin{abstract}
    This chapter explores various techniques you can use to integrate stuffs. You still have to keep in mind that even though this chapter is very abstracted away, \emph{integrals still have its geometric meaning: area under the graph.} This geometric interpretation is going to somewhat appear quite often throughout the chapter, so be aware of it.

    Also, some integrals are definite: they have clear boundaries of integrations. You also have to keep track of what variable is bounded by that bound, otherwise you might end up evaluating incorrectly.
\end{abstract}

Note from the writer: As a calculus teacher myself, I usually don't teach this topic but leave it as an exercise. Techniques of integrations come naturally with much experiences. Therefore, I list a table of integrations at the end of the chapter for the physicists out there.

\section{Change of variables}
\label{sec:changeofvariables}

Change of variables, or integration by substitution is a method to convert an unknown integrals into a more familiar one. Consider $\int x\e^(x^2)\dd{x}$. This integral might seem daunting at first but, notice that the derivative of $x^2$ is exactly $x$. I shall introduce a new variable $u$ where $u = x^2$. If we take the derivative w.r.t. $x$ on both sides,
\begin{align*}
    \dv{u}{x} = 2x \\
    \dv{u} = 2x\dd{x}.
\end{align*}

\section{Integrals of inverse functions}

\section{Integrals of symmetric functions}

\subsection{Even and odd functions}

\subsection{Functions with axial and spherical symmetry}

\section{Integration by part}

\section{Recurrence relations and reduction formulas}

\section{Working with complex numbers and exponentials}

\section{Trigonometric substitution}

\section{Rational functions}

\subsection{Partial fraction decomposition}

\subsection{Quadratic under radicals}

\subsection{Ostrogradski method}

\section{Feynman's trick for integration}

\section{Polar integration}

\section{Gaussian integral}

\section{Cauchy's formula for repeated integrations}

\section{Numerical integration}